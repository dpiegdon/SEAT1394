\documentclass{beamer}


\mode<presentation>
{
  \usetheme[compress]{Ilmenau}
	\useinnertheme{circles}
	\usecolortheme{lostrace}
  \setbeamercovered{transparent}
  \setbeamertemplate{navigation symbols}{}
}

%\usepackage{epsfig,graphicx,epsf}
\usepackage[ngerman]{babel}
\usepackage{ucs}
\usepackage[utf8x]{inputenc}
\usepackage{times}
\usepackage[T1]{fontenc}
\usepackage{multimedia}

\newenvironment{itemizeframe}[1]
  {\begin{frame}{#1}\startitemizeframe}
  {\stopitemizeframe\end{frame}}
\newcommand\startitemizeframe{\begin{itemize}}
\newcommand\stopitemizeframe{\end{itemize}}



\title[hacking in physically addressable memory]
{ { \small Seminar of Advanced Exploitation Techniques, WS 2006/2007} \\ \textbf{hacking in physically addressable memory}\\ a proof of concept}

\author[losTrace A.K.A. David R. Piegdon <david.rasmus.piegdon@rwth-aachen.de>]
{David~Rasmus~Piegdon \\ \ \\ \tiny Supervisor: Lexi Pimenidis}

\institute[RWTH Aachen University of Technology]{
	Lehrstuhl f\"ur Informatik IV, RWTH Aachen\\ {\tiny \ \\ http://www-i4.informatik.rwth-aachen.de \\} 
}

\date[2007-02-21]
{February 21st 2006}

\subject{Direct Memory Access hacking}

%\pgfdeclareimage[height=0.4cm]{by-nc-sa}{../bilder/by-nc-sa}
%\titlegraphic{\href{http://creativecommons.org/licenses/by-nc-sa/2.0/de/}{\pgfuseimage{by-nc-sa}}}

\pgfdeclareimage[height=1.0cm]{i4-1394}{i4_ieee1394_stacked}
\logo{\pgfuseimage{i4-1394}}

%\pgfdeclareimage[height=3cm]{microkernel}{microkernel}
%\pgfdeclareimage[height=6cm]{monolith}{monolith}
%\pgfdeclareimage[height=19cm]{kernel-attackspace}{kernel-attackspace}


\begin{document}

\begin{frame}
	\titlepage
\end{frame}

\begin{frame}
	\frametitle{Gliederung}
	\tableofcontents[hideallsubsections]
\end{frame}

\AtBeginSection{
%	\begin{frame}
%	  \frametitle{Gliederung}
%	  \tableofcontents[hideallsubsections]
%	  % Die Option [pausesections] könnte nützlich sein.
%	\end{frame}
	\begin{frame}<beamer>
		\frametitle{Gliederung}
		\tableofcontents[current,hideallsubsections]
	\end{frame}
}

% "` "'
% umlaute "a ...

% \begin{exampleblock}	gruene box
% \begin{alertblock}	rote box
% \begin{Definition}	blaue box, titel "Definition"
% \begin{Example}	gruene box, titel "Example"
% \alert{...}		roter text
% \structure{...}	blauer text

\section{Introduction}

	\subsection{physical addressable memory}

		\begin{frame}{physical addressable memory}
			"`hacking in physically addressable memory"'
			\begin{itemize}
				\item hacking: using a technique for something it has not been designed for.
				\item physically addressable memory: direct memory access, "`DMA"'
			\end{itemize}
		\end{frame}

		\begin{itemizeframe}{hacking}
			\item I will show mostly \structure{attacks}
			\item so actually I will be \structure{cracking} a systems security
			\item<2-> \alert{exploiting et al is not hacking by definition}
			\item<2-> \structure{hackers hate misuse of "`to hack"' by media}
		\end{itemizeframe}

		\begin{itemizeframe}{DMA}
			\item DMA = Direct Memory Access
			\item basic requirement for introduced approach
			\item known for a long time: attacker has DMA -> \structure{0wn3d}
				\begin{itemize}
					\item 0wn3d by an iPod $[1]$
					\item and others $[2,3]$
				\end{itemize}
			\item this is a \structure{proof of concept}
		\end{itemizeframe}
	
		\begin{itemizeframe}{Proof of Concept}
			\item<1-> Accessing phys. addr. memory
			\item<2-> Finding virtual address spaces
			\item<3-> \structure{Gathering information}
			\item<4-> \structure{Injecting code}
			\item<5-> Prospects
		\end{itemizeframe}

\section*{}

	\begin{frame}
		Fragen?
	\end{frame}

	\begin{itemizeframe}{Literatur}
		\item[1] \emph{Michael Becher, Maximillian Dornseif, and Christian N.
			Klein.} \structure{Firewire - all your memory are belong to us}, 2005.

		\item[2] \emph{Adam Boileau.} Ruxcon 2006: Hit by a bus: Physical access attacks with
			firewire, 2006.

		\item[3] \emph{Mariusz Burdach.} Finding digital evidence in physical memory, 2006.
	\end{itemizeframe}

	\begin{itemizeframe}{Literatur}
		\item[4] \emph{Rudi Cilibrasi and Paul M. B. Vit\'anyi.} Clustering by compression.
			IEEE transactions on information theory, vol. 51, 2005.

		\item[5] \emph{Otto Spaniol et al.} Systemprogrammierung, Skript zur Vorlesung an der RWTH
			Aachen. Wissenschaftsverlag Mainz; Aachener Beitraege zur Informatik (ABI),
			2002. ISBN 3-86073-470-9.
			
		\item[6] \emph{Intel Corp.} Intel 64 and IA-32 Architectures Software Developer’s Manual.

		\item[7] \emph{Bruce Schneier.} Applied Cryptography (Second Edition). John Wiley \& Sons,
			Inc, 1996. ISBN 0-471-11709-9.
	\end{itemizeframe}

\end{document}

