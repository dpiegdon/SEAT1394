% vim: tw=80 ai fdl=99 fo+=a
%
% $Id: attacks.tex 349 2007-02-04 15:43:37Z lostrace $
%

\section{Attacking}

\label{attacks}

In this section we will show how the two libraries discussed in sections
\ref{memsources} and \ref{addresstranslations} can be used to gain unauthorized
access to remote computer systems.

\textbf{Subsection \ref{information_gathering}} will discuss passive attacks
that only \emph{read} from a physical memory source. An overview over gathering
information from a virtual address space is given, including finding processes
on a host, obtaining their environment, arguments and the path of the processes
binary.  Afterwards, an additional attack is introduced that is capable of
copying public/private keypairs for SSH from a running \texttt{ssh-agent}
process.

In \textbf{subsection \ref{userspace_attacks}}, we will introduce attacks that
change data in the virtual address space. Further details about executables,
libraries and processes will be given as an introduction. Then we will show how
to find mapped libraries, binaries and the stack of a process and how to inject
code into a running process. We then introduce a specially crafted code that can
be used to obtain an interactive shell using firewire access only.

\subsection{Information Gathering}

\label{information_gathering}

\subsubsection{Identifying processes}

\label{identifying_processes}  Once an address translation table has been found,
it is of interest, what kind of process resides in this virtual address space.
For userspace applications on IA-32-Linux there is a simple way to identify a
process's \emph{filename}, its \emph{arguments} and even its full set of
\emph{environment variables}: This information is often required by a process
and thus the kernel will provide it to the process by copying it to the bottom
pages of the application's stack\footnote{i.e.  the stack-pages that are found
first when seeking downward from virtual address \texttt{0xbfff~f000}}.

We wrote a tool \texttt{remote-ps}, that uses a function \texttt{proc\_info()}
to seek the stack-bottom, parse it and return ready-for-use environment vectors,
command-line vectors and the full path of the binary for a given linear address
space.  For each found address space it will print a list of all found processes
with their arguments.

% FIXME: example screen shot

% FIXME ``etwas mehr drueber, was man damit alles boeses machen kann''

\subsubsection{Places to find secrets}

Many applications keep confidential data in their memory, some of them even
locking them into the main memory\footnote{e.g. via the \emph{mlock} function}
to prevent the operating system from swapping them to slower (permanent) media.
While in general this is a good idea, as an attacker may reconstruct the data
from e.g.~thrown-away harddisks, it increases the chance of an attacker that can
obtain access to the memory of the system in question, as the confidential
material will be stored in memory completely and unfragmented.

``Secrets'' includes, among other information, \emph{authentication data},
\emph{cryptographic key material}, \emph{random data} (e.g. to seed a
cryptographic algorithm) and sometimes even \emph{algorithms} (\emph{proprietary
software}). Authentication data can be e.g.~passwords or private keys for
signature algorithms. Cryptographic key material, are keys for usage with
cryptographic algorithms (like signature algorithms). The latter two will be of
main interest in the remainder of this section.

Many applications using a cryptographic infrastructure for communications will
keep once loaded passwords or keys in their main memory for successive usage.
The operating systems protection model ensures the safety of this information
from other processes running on the same system; but by accessing the main
memory we do have full access to this material. The only remaining task is to
find and reconstruct the key material and passwords from the memory.

As an example, the following applications are of interest:

\begin{itemize}

	\item GnuPG and PGP: applications to sign and encrypt arbitrary data
	with public/private keypairs. They are wide spread for email-encryption
	and -signing.

	\item sshd, ssh and ssh-agent: the \emph{secure shell} application is an
	extended, encrypted version of telnet using strong cryptography,
	including passwords, skey, x509 certificates, RSA and DSA keys.

	\item Apache and other SSL-enabled web servers.

	\item OpenVPN, Cisco-VPN and other VPN-servers and clients

	\item Instant Messaging Applications, e.g.~Psi, keeps the authentication
	information and possibly the GnuPG keypair in memory.

	\item The computer BIOS password, ATA password or PGP-Wholedisk
		password: the computer or its drives can be locked with a BIOS
		password or the harddisk can be encrypted.  For a sample attack,
		see \cite{rux2k6firewire:2006}.

\end{itemize}


\subsubsection{Example attack: ssh-agent snarfer}

\label{ssh-agent-snarfer} To show how easy it is to obtain secret keys from a
process we have written a sample attack to obtain
%
(snarf\footnote{to snarf: To grab, esp. to grab a large document or file for the
purpose of using it with or without the author's permission. // To acquire, with
little concern for legal forms or politesse (but not quite by stealing).
(source: \href{http://catb.org/jargon/html/S/snarf.html}{Jargon Files})})
%
\emph{ssh public/private keypairs} from \texttt{ssh-agent}s via \emph{firewire}.

When using \texttt{ssh} for accessing remote computers it is possible to
authenticate via passwords, public/private keypairs and various other methods.
The usage of public/private keypairs is wide-spread among people using
\texttt{ssh} on a regular basis. These keypairs can either be a DSA or a RSA
keypair, they are typically created with \texttt{ssh-keygen} and stored
somewhere in \texttt{\$HOME/.ssh/}, e.g.~\texttt{/root/.ssh/id\_dsa} and
\texttt{/root/.ssh/id\_dsa.pub}. Keypairs can and should be encrypted with a
passphrase to prevent attackers from using them, if they were able to obtain
them by some means. Thus to use a keypair it is required to enter this
passphrase each time.  This can be disturbing during frequent usage, e.g.~when
using \texttt{ssh+svn} or \texttt{scp} with remote-tab-completion (\texttt{zsh}
is capable of this).

For these and other reasons, the \texttt{ssh-agent} has been developed. This
agent will run in the background; the user can store a keypair into it (once
entering the passphrase to unlock the keypair) and successively use the keypair
without the requirement to enter the passphrase each time. The keypair can be
wiped from memory on demand and also be loaded only for a specified period of
time.

%Actually, here begins the fun.
During our tests we found that the key is \emph{not} wiped from memory when the
time limit is hit. It will be wiped the next time the ssh-agent is queried (via
its socket), but the agent is stalled in a \emph{read} system call until this
query and thus can not wipe the key. That makes it possible to obtain long
overdue keys from \texttt{ssh-agent}s, although their owners believed them to be
safe. A simple timer could have prevented this\footnote{We hereby strongly
encourage the developers to implement such a timer!}. But even with such a timer
enabled it would be possible to acquire the key during its lifetime.

To obtain a keypair from an agent via firewire, a staged attack is required:



\begin{enumerate}

	\item Seek the first GiB of physical memory for pagetables.

	\item For each pagetable: check with the introduced
		\texttt{proc\_info()}, if the found userspace belongs to a
		\texttt{ssh-agent} process. If not, seek next pagetable.

	\item Use the obtained environment to resolve the users home directory
		(\texttt{\$HOME}) and create a path where keypairs most likely
		reside in the file system (e.g.  ``\texttt{\$HOME/.ssh/}'') and
		seek this string in the heap.  This approach will only find
		keypairs that have been loaded with this
		key-location.\footnote{Actually this field is the key's
		comment-field that is mostly unused and overwritten with the
		filename of the key. Keypairs that are used with SSH protocol
		version 2 (virtually all) do not have a comment-field; during
		loading, the comment-field is always initialized with the keys
		pathname.} Keypairs loaded from different locations or via a
		relative path can thus not be found by this search.
	
	\item All loaded keypairs have a corresponding \emph{identity-struct} in
		an agent (see figure \ref{fig:code:identity-struct}). Among
		other fields, this identity-struct contains a link to a key
		struct, the above mentioned path/comment-field and the lifetime
		of the key. Thus to find the identity struct corresponding to a
		found comment-field, one has to search the address of the
		comment-field in the heap of the agent.

\lstset{language=C, numbers=left, numberstyle=\tiny, frame=lines}
\begin{figure}[h] \begin{center}
	\tiny
	\begin{lstlisting}
		typedef struct identity {
			TAILQ_ENTRY(identity) next;
			Key *key;
			char *comment;
			u_int death;
			u_int confirm;
		} Identity;

		typedef struct {
			int nentries;
			TAILQ_HEAD(idqueue, identity) idlist;
		} Idtab;

		/* private key table, one per protocol version */
		Idtab idtable[3];
	\end{lstlisting}
	\caption{\texttt{openssh/ssh-agent.c}: Identity structure and \texttt{idtable}}
	\label{fig:code:identity-struct}
\end{center}\end{figure}
	
	\item Once the \emph{key-struct} (figure \ref{fig:code:key-struct}),
		that is linked to by the identity-struct, has been found, one
		can determine whether the found key is a RSA or a DSA key.  The
		key-struct contains a type-field and two pointers to either the
		RSA or the DSA key.  These referenced structures are the
		\emph{OpenSSL}\footnote{OpenSSL
		(\href{http://openssl.org}{http://openssl.org}) is a free
		open-source implementation of the secure socket layer protocol
		also providing a general purpose cryptography library
		(\text{libcrypto}).}-structures \texttt{RSA} and \texttt{DSA}.
	
\lstset{language=C, numbers=left, numberstyle=\tiny, frame=lines}
\begin{figure}[h] \begin{center}
	\tiny
	\begin{lstlisting}
		enum types {
			KEY_RSA1,
			KEY_RSA,
			KEY_DSA,
			KEY_UNSPEC
		};

		struct Key {
			int      type;
			int      flags;
			RSA     *rsa;
			DSA     *dsa;
		};
	\end{lstlisting}
	\caption{\texttt{openssh/key.h}: Key-structure}
	\label{fig:code:key-struct}
\end{center}\end{figure}

	\item For both RSA and DSA structures (figure \ref{fig:code:rsa-struct}
		and \ref{fig:code:dsa-struct}), all important fields need to be recovered to
		obtain valid keypairs.
		\cite{applied_crypto:1996,handbook_applied_crypto:2001} give an overview of
		both cryptographic algorithms, \cite{openssl_book:2002} introduces OpenSSL
		concepts and implementation details. OpenSSL`s arbitrary precision integer
		implementation is the \texttt{BIGNUM}-struct (often abbreviated ``BN'').  It
		consists of a variable-length array of bit-vectors forming the value and a
		length-field defining the length of this array (see figure
		\ref{fig:code:bignum-struct}).  As RSA and DSA both operate on finite
		fields, both are implemented with \texttt{BIGNUM}s.  Therefore, the RSA and
		DSA structures contain several \texttt{BIGNUM}s that need to be recovered to
		obtain a valid copy of the keypair.
	
\lstset{language=C, numbers=left, numberstyle=\tiny, frame=lines}
\begin{figure}[h] \begin{center}
	\tiny
	\begin{lstlisting}
		/* Declared already in ossl_typ.h */
		/* typedef struct rsa_st RSA; */
		struct rsa_st {
			int pad;
			long version;
			const RSA_METHOD *meth;
			ENGINE *engine;
			BIGNUM *n;
			BIGNUM *e;
			BIGNUM *d;
			BIGNUM *p;
			BIGNUM *q;
			BIGNUM *dmp1;
			BIGNUM *dmq1;
			BIGNUM *iqmp;
			int flags;
		};
	\end{lstlisting}
	\caption{\texttt{openssl/crypto/rsa/rsa.h}: RSA structure (stripped down)}
	\label{fig:code:rsa-struct}
\end{center}\end{figure}

\lstset{language=C, numbers=left, numberstyle=\tiny, frame=lines}
\begin{figure}[h] \begin{center}
	\tiny
	\begin{lstlisting}
		/* Already defined in ossl_typ.h */
		/* typedef struct dsa_st DSA; */
		struct dsa_st {
			int pad;
			long version;
			int write_params;
			BIGNUM *p;
			BIGNUM *q;      /* == 20 */
			BIGNUM *g;
			BIGNUM *pub_key;  /* y public key */
			BIGNUM *priv_key; /* x private key */
			int flags;
			const DSA_METHOD *meth;
			ENGINE *engine;
		};
	\end{lstlisting}
	\caption{\texttt{openssl/crypto/dsa/dsa.h}: DSA structure (stripped down)}
	\label{fig:code:dsa-struct}
\end{center}\end{figure}

\lstset{language=C, numbers=left, numberstyle=\tiny, frame=lines}
\begin{figure}[h] \begin{center}
	\tiny
	\begin{lstlisting}
		/* Already declared in ossl_typ.h */
		/* typedef struct bignum_st BIGNUM; */
		struct bignum_st {
			BN_ULONG *d;    /* Pointer to an array of 'BN_BITS2' bit chunks. */
			int top;        /* Index of last used d +1. */
			/* The next are internal book keeping for bn_expand. */
			int dmax;       /* Size of the d array. */
			int neg;        /* one if the number is negative */
			int flags;
		};
	\end{lstlisting}
	\caption{\texttt{openssl/crypto/bn/bn.h}: BIGNUM structure (stripped down)}
	\label{fig:code:bignum-struct}
\end{center}\end{figure}

	\item Some validity tests may be done to verify that the acquired
	\texttt{BIGNUM}s fulfill algorithm-specific properties and thus form a valid
	keypair.

	\item Attach the obtained \texttt{BIGNUM}s back into valid RSA or DSA
		structures and save these keys to a file using
		openssl-functions.
	
\end{enumerate}




%As seen, the search algorithm (2,3,4) has its downsides but works astonishingly
%well\footnote{To create a better algorithm remains as an exercise to the
%interested reader.}. A much better algorithm to find the \emph{identity}-structs
%is to use the ELF-headers of the mapped executable to resolve the symbol of the
%\emph{idtable}\footnote{The \emph{idtable} is a structure referencing all keys
%that are loaded into the agent.}. This approach is straight-forward, hits
%\emph{all} keys and should work almost always. The downside of this approach is
%that most distributions distribute programs with their symbols stripped (due to
%size and security reasons); this invalidates the symbol-resolution-approach, as
%this stripping also removes any information of the \emph{idtable} symbol.

Once \emph{one} \emph{identity}-struct is found, \emph{all} structs of the same
key-type (RSA or DSA) could be found by walking the list this key is linked
into.

As stated above, the keypairs reside decrypted in the memory of the agent (even
if overtime) and thus, when snarfed and stored to a file, can be immediately
used by the command \texttt{ssh~-i~keyfile~user@host}\footnote{Only by stealing
a key, an attacker will not know, which hosts can be accessed with a retrieved
key.}.  Such an attack will not take much longer than searching the first 1 GiB
of physical RAM for pagedirectories, that is \emph{typically no more than 15
seconds}. If an attack fails but an agent was found, it would be possible to
just dump the heap of the agent to a permanent storage and stage a more thorough
attack at a later time. Once the heap is dumped, all required data is obtained.
A similar attack via \emph{ptrace} should be possible as well.

%The reader may refer to \texttt{attacks/information/snarf-sshkey.c} in the
%corresponding tarball for the source-code of the attack. Please keep in mind
%that this attack will only find keys loaded with the absolute path
%\texttt{\$HOME/.ssh/}.


%\subsubsection{Matching and statistics to find secret keys}

\cite{hide_n_seek:1998} introduces some schemes to find secret keys in random
data and some countermeasures. It takes a special look at finding private RSA
keys if their corresponding public keys are known and finding keys by searching
high-entropy regions. Though we encourage the reader to read this interesting
paper, the circumstances are most likely very different: cheap and small storage
media like flash-memory and small harddisks have increased portable storage to a
huge size, equal or larger than the memory a computer system typically has.
Thus, an attacker can just dump the full memory or a subset of it (like the
virtual address space of a single process) that is promising to contain a
secret. A thorough attack can then be staged later.  Still, searching private
keys with the introduced methods can be very helpful, if reconstruction of the
used data-structures is impossible or more expensive. Furthermore, when trying
to obtain a private key, often enough the corresponding public key is unknown.
This invalidates the approaches introduced to find RSA secret keys that require
the public key.




\subsection{Userspace Modifications}

\label{userspace_attacks}

Each executable object, including libraries and executables, is usually
separated into code and data sections, where the code is marked as read-only
during execution. The data can further be separated into read-only data,
read-write data and uninitialized global variables (local variables will be
allocated from the stack, runtime-allocated memory is allocated from the heap).
Thus an executable object may be split into four regions: \emph{code},
\emph{rodata} (constants), \emph{rwdata} and \emph{dynamic data} (rwdata may
also be implemented by copying the initial data from rodata to dynamic data);
stack and heap are process-specific.

As many different processes may use the same executable objects, it would be a
waste of memory if the operating system created a copy of the object for each
reference to it.

Code and rodata may not be written to by a process, thus the operating system
can share these two regions among processes that are using the same objects
(binaries or libraries).  Thus, once an operating system \emph{ensures} that a
process can not write to code-regions and read-only data-regions or introduces a
copy-on-write mechanism, it can map these once loaded regions into multiple
virtual address spaces.  This enforcement is done in hardware, on IA-32 by
setting a flag in the page directory or a page table referencing the specific
physical memory pages containing the region.

Newer CPUs provide \emph{page-level no-execute enforcements} (AMD's NX \emph{No
eXecute bit}, Intel's EXB \emph{EXecute disable Bit}); equal segment-level
enforcements exist for years but never have been used in mainstream as Linux and
many other operating-systems using a flat memory-model (with only one big
segment spanning the full virtual address space).  Once these page-level
enforcements are used in systems, attacks that inject code into data-regions or
the stack are rendered completely useless. However with access to the
data-structures (page directory and page tables) containing the information,
which pages are executable and which not, it is trivial to remove the protection
before injecting code e.g.~into an applications stack. 

On Linux, programs and libraries are in \emph{Executable and Linker Format}
(ELF).  This format is described in the manpage \texttt{elf(5)}. When a binary
is mapped into a process's memory, it is mapped including the full ELF header
containing all information that is required to link all references between
different objects; ELFs are always mapped at page-bounds. Due to this, all
mapped ELFs (that includes executables and libraries) can be found by scanning
all pages for the ELF Magic (0x7f E L F) at offset $0$ in the page.  Libraries,
executables and other ELF objects can be distinguished by evaluating the
\texttt{e\_type} field of the ELF header.

\subsubsection{Overwriting executable or library code}

When code of executables or libraries is changed, all programs using these ELF
objects are influenced at the same time. An attacker thus has the ability, but
also the burden, to possibly infect several processes at the same time. Such an
attack has to be carefully prepared and conducted, as each system may have a
different version of a binary and overwriting the wrong parts of an ELF or
writing the wrong code may result in an almost immediate crash of all processes
using the ELF. Though this is an interesting approach, there are easier ways to
inject code into a \emph{single} process (see \ref{overwriting_stack}).

Such an attack could be conducted by searching a single virtual address space
for the glibc and then parsing the ELF-headers and searching for the entry-point
of the \texttt{printf}-function (or some other function).  Then a piece of code
could be injected into the unused fragment of the last page of the libc-mapping.
It is important to inject the shellcode into the mapping, as all processes that
will be affected have to be able to reach the shellcode.  The intention is to
overwrite the \texttt{printf} functions code with a \emph{relative}\footnote{An
ELF may be mapped at different locations in different processes, if it is
``PIC'' or ``PIE'' (Position Independent Code/Executable) and the kernel
supports this.  Thus, unless the ELF is only mapped in one process or the
overwritten function is only used in one process, the jumps target has to be
addressed relative to the current position.} jump to this injected code.  But as
we need to overwrite some instructions inside the function, we need to parse the
functions code to separate each instruction\footnote{on IA-32, different machine
instructions can have different length}, so that after our code is executed, it
executes a copy of the overwritten instructions and jumps to a fully intact
instruction right after the injected jump. After this has been done, the jump
can be injected into the functions code. This last write has to be as atomic as
possible, as a process may just be executing these bytes and thus get astray.
On IA-32, entry-points of functions are most likely aligned to 32-bit
addresses\footnote{due to optimizations by the compiler}.  Firewire also
provides an interface to write 32-bit aligned 32-bit values (``quads'')
atomically.  Unfortunately, a \emph{relative short} jump ($2$ bytes) can only
jump within $\pm256$ bytes from the jump itself and \emph{relative long} and
\emph{absolute} jumps are $5$ bytes wide ($1$ byte command + $4$ bytes address).
A short jump is most likely incapable of reaching the last page of the ELF and
writing a long relative jump is not atomic.

A lot of interesting methods to inject code into a running process have been
developed; e.g.  \cite{phrack59.8:2002} gives an introduction into using
\emph{ptrace}, including injecting whole shared objects using the runtime linker
\emph{libdl}.  The usability of this approach has not yet been analysed.


\subsubsection{Overwriting the stack and return addresses}

\label{overwriting_stack} Besides stealing SSH-keys, we have put most of our
efforts into injecting code into the stack and overwriting return-addresses on
the stack to point to the injected code. This modification of the classic
\emph{stack overflow} method has some advantages over the previous approach:

\begin{itemize}

	\item Each process, even each thread, has its own stack. Thus only a
		single thread will be affected by the attack.

	\item If the attack fails and the thread dies, only a \emph{single}
		thread will fail on the target system, not e.g. \emph{all}
		processes using the glibc or \emph{all} instances of $/bin/sh$.

	\item The process can read and write to the stack as well, thus we can
		communicate with the injected shell-code in a rather easy way
		(see \ref{communication_DMA}).

	\item During the attack we do not need to modify parts of the code of
		the target-process, reducing the risk of an astray process. The
		final part consists of overwriting 4 byte wide return-addresses
		on the stack and this can almost always be done automatically.

\end{itemize}

The attack consists of the following steps:

\begin{enumerate}

	\item Search a free location in the stack-pages. If the shellcode is
		small, we can use the zero-padded area of the pages containing
		the environment and argument vectors (see section
		\ref{identifying_processes}). If the shellcode does not fit into
		this area, we could try to just overwrite these vectors. Most
		programs will parse environment and arguments only once during startup of
		the program, thus overwriting them at a later time usually has no effects.
		But note that these vectors are also evaluated whenever someone accesses a
		processes \emph{procfs} entries \texttt{/proc/\$PID/environ} and
		\texttt{/proc/\$PID/cmdline}. Thus if these are overwritten with new data,
		it is possible to see the difference by querying these procfs-entries or
		using \texttt{ps}\footnote{\texttt{ps} relies on these procfs entries}.


	\item Scan the stack for stack-frames and for each found: overwrite the
		return address. This can be simplified into: overwrite all
		32-bit aligned 32-bit values that contain a value that might be
		a pointer into the main code area (\texttt{0x08**~****}, see
		figure \ref{fig:linux_virtual_address_space}) with a pointer to
		the injected code.  A more aggressive approach might also
		overwrite return-values pointing into the library section
		(\texttt{0xB7**~****}).

\end{enumerate}

Once the attacked process leaves a stack frame with an overwritten return value,
it will jump to the injected code and execute it.

An implementation of this attack, including some sample shellcodes like a
bindshell and a simple \texttt{printf} has been implemented. %can be found in
%\texttt{attacks/userspace/inject-code.c} and \linebreak
%\texttt{attacks/userspace/shellcodes/}.

% FIXME: show screen dump here!

% FIXME: mehr infos zu ``Identifying processes'', danach hier darauf aufbauen?
% Mehr infos zu stack-frames?

\subsubsection{Direct communication with shellcode via DMA}

% FIXME: masterpiece / royal league?

\label{communication_DMA} The royal league of attacking remote hosts is to
inject code that is executed and spawns an interactive shell (thus it is also
called ``shellcode'') and additionally is as invisible on the target system as
possible.  An interactive shell has to communicate with its user, so typically a
network-based shell is used for attacking purposes. The downside of this
technique is the visibility of the communication on the network-layer: an
administrator can easily spot the network connection by either sniffing on the
network or by asking the system what kind of sockets and files a process is
using\footnote{e.g. by using \texttt{lsof -i} (LiSt Open Files)}. A network
intrusion system (NIS) can easily spot shell connections in an automated way or
firewalls could be configured in a way that a network connection is impossible.
But when using e.g.~firewire to attack a host there is usually no network
connection between the attacker and the victim at all.

In our example attack, we will use a similar attack vector to inject a
``beachhead'' with our methods of physical memory access.

The overall mechanism is introduced in figure \ref{fig:functionality_beachhead}.
The injected shellcode will fork a shell and communicate with stdin/stdout of
the shell via two pipes. The shellcode then creates a second thread, thus having
one thread for each direction of \emph{master to shell} and \emph{shell to
master}. If the master (attacker) wishes to send a command to the shell, it
writes the command string into the \emph{FromMaster} ring-buffer via DMA.  Once
the ReaderThread sees that the ring-buffer is not empty, it reads the data
inside the buffer and writes it into the pipe to the shell.  The WriterThread
will read data coming from the shell from the pipe and then write it into the
\emph{ToMaster} ring-buffer, so the master can read it via DMA.

The shellcode of the beachhead
%(\texttt{attacks/userspace/shellcodes/dmashellcode.s}) 
is introduced in appendix \ref{beachhead_pseudocode} in pseudo-code. 
%The program to inject the shellcode and communicate with it is
%\texttt{attacks/userspace/dmashell.c}.

\begin{figure}[htb] \begin{center}

	\epsfysize 4.5cm

	\epsffile{functionality_beachhead.eps}

	\caption{Functionality of the beachhead}

	\label{fig:functionality_beachhead}

\end{center}\end{figure}

% FIXME: screen shot of root shell?

We thus have shown that it is possible to gain interactive, unauthorized,
administrative access to a remote computer system by simply plugging a cable
into the device's firewire port.

