% vim: tw=80 ai fdl=99 fo+=a
%
% $Id$
%

\section{Introduction}

All modern operating systems do not grant processes and users access to
physically addressed memory, as this addressing mode circumvents any protection
methods provided by virtual addressing to separate processes from each other and
the operating system. Only the operating system may use physical addressing to
prepare address spaces for each running process, manage these, access special
memory of extension cards and alike, or even only during bootstrapping as Linux
does. Having access to a computer's memory is equal to have the same rights and
possibilities as the operating system. Thus access to it should require system
administrator rights or physical access to the hardware of the underlying
system.  Therefore it is crucial for a system's security to prevent attackers
from gaining direct access to a computer's memory.

Up to recently, protecting access to a computer's memory was equal to defend
against physical attacks on the hardware, given that the operating system had no
vulnerabilities.  Thus, reading and writing to a computer's memory was only
possible by booting custom operating system, opening the case and attacking the
hardware directly, stealing the whole system, installing specially crafted
PCMCIA cards, or the like.  

However, IEEE1394, also known as ``firewire'' and so called in the rest of the
paper, does not require to boot a custom operating system, to open the case,
steal parts of the hardware, install new hardware (except plugging in the
firewire cable) or specially crafted hardware.  Access via firewire is as easy
as plugging in a firewire device, like e.g.~an iPod, letting it do its job and
unplugging it.

In this paper we will introduce several advances in attacking computers via
firewire.  As a foundation for the attacks, we introduce two libraries that are
used as a step stone for further work. These libraries are used to access
virtual address spaces of any process on the victim's host. They also provide a
simple, generic interface for all kinds of physical memory sources. As an
example, we implemented backends for IEEE1394 and filedescriptors so far, but
other sources can be trivially added.

Once, access to the physical memory of a system is obtained, there are two
obvious ways to extract useful information from it: 
\begin{itemize}

	\item It is possible to parse the operating systems internal data
	structures holding all relevant information about loaded drivers,
	running processes et al.
	
	\item It is possible to use the information that the operating system
	provides to the hardware to tell it about the virtual address spaces of
	each process.
	
\end{itemize}

The first scenario will not work between different operating systems and
architectures, since it is necessary to write a parser for each combination of
them, possibly even for different versions of the same operating system. 

The latter uses an information structure that only changes between different
architectures, as the architecture relies on it. Furthermore there is a well
defined algorithm for using this information (implemented in hardware in the
architecture, but well defined in the reference manuals for this architecture,
so system designers can provide valid data to the hardware). On the other hand,
the second approach does not give as much information about the system as the
first, since the first obtains all information directly from the kernel
structures, while using the second approach we only can enter virtual address
spaces of processes. However, in the following we will use the second approach
for most attacks, as it is more robust.

In \cite{finding_digital_evidence_in_physical_memory:2006}, an approach is
introduced that parses kernel-structures of Windows and Linux kernels. Since the
paper is about \emph{finding an attacker} and not \emph{attacking a system}, it
can be savely assumed that the architecture, the operating system and its
version are known. An attacker, on the other hand, is usually left with guessing
the architecture and operating system and needs more robust tools for his
attacks.% (Obviously this is only a short-term
%argument, as an attacker can also write such tools for \emph{all} OS version
%and architecture combinations\ldots but the heck with this...).  FIXME ``the
%heck with this''... spida: ``for simplicities sake, I will do this for only one
%one combination here, as this is intended to be only a proof of concept''

Latest state of the art was shown in \cite{cansecwest_firewire:2005}, where
Dornseif et al.~demonstrated how to connect to a remote computer over an
IEEE1394 connection and changed the user of all running processes to the super
user ``root''. However, their work left the victim's machine in a fragile state,
often prone to crashing, and without direct gain for the attacker.

Other advances in memory analysis have been introduced in
\cite{finding_digital_evidence_in_physical_memory:2006}, where kernel-structures
are searched and parsed to identify processes and meta-in\-for\-ma\-tion about
processes and the system itself.

Our contributions in to this area are twofold: 

\begin{itemize} 

\item We demonstrate how to access the memory of a remote machine in a
structured and portable manner, making information retrieval as easy as reading
local memory.

\item With the help of our tool-set we demonstrate that it is not only possible
to read or change data in the remote machine, but also to execute code and
obtain interactive access, possibly with superuser privileges.

\end{itemize}

\subsection{Roadmap}

In \textbf{Section \ref{memsources}}, we will introduce \texttt{libphysical}, a
library providing an attacker with a simple, generic interface to interact with
physical memory via a simple interface.  Thereafter, in \textbf{Section
\ref{addresstranslations}}, \texttt{liblinear}, an interface to access virtual
address spaces, will be introduced. It incorporates a backend for
IA-32\footnote{This backend is missing algorithms for less-used operation-modes
of the IA-32 architecture, but it will work at least for most kinds of the
\emph{Linux} kernels ($\leq$ 4GB RAM). It has not yet been tested with
\emph{Windows} or \emph{MacOS X} and is missing features (Virtual-8086 mode) to
work with DOS-processes running inside Windows.} and functions to find virtual
address spaces.  We introduce several attacks basing on these results in
\textbf{Section \ref{attacks}}, ranging from simple information gathering up to
obtaining an interactive shell.

In \textbf{Section \ref{prospects}}, prospects will be given, what further kinds
of attacks seem to be possible and are of interest. The paper ends with
conclusions in \textbf{Section \ref{conclusion}}.

