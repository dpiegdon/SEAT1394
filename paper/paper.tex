% vim: tw=80 ai fdl=99 fo+=a
%
% $Id$
%

\documentclass[twoside,a4paper,graphics,11pt,dvips]{article}

\usepackage{graphicx,curves,epsf,float,rotating}
\usepackage{color}
\usepackage{epsf,epsfig,eepic}
\usepackage{stmaryrd,latexsym}
\usepackage{a4,amsmath,amsfonts,amssymb}
\usepackage{abstract}

\usepackage{listings}

%\usepackage{endnotes}
%\let\footnote=\endnote

\textwidth 15cm
\textheight 23cm
\oddsidemargin 1cm
\evensidemargin 0cm

\author{David R. Piegdon}
%\date{\today}
\date{April 12, 2007}
\title{hacking in physically addressable memory}
%\thanks{thankthankthank}

\hyphenation{me-cha-nisms me-mo-ry for-ming ty-pi-cal-ly tho-rough ope-ra-ting
phy-si-cal-ly ad-dres-sing se-cond ma-na-ge-ment si-mi-la-ri-ty re-co-ver-ed
eXe-cute in-ject-ing}

% according to lshort, hyperref has to be last in preamble
\usepackage{hyperref}
% linkcolor defaults:
% 
% linkcolor   red	     Color for normal internal links.
% anchorcolor black    Color for anchor text.
% citecolor   green    Color for bibligraphical citations in text.
% filecolor   magenta  Color for URLs which open local files.
% menucolor   red	     Color for Acrobat menu items.
% pagecolor   red	     Color for links to other pages.
% urlcolor    cyan     Color for linked URLs. 
\hypersetup{colorlinks}
\hypersetup{linkcolor=blue}
\hypersetup{urlcolor=blue}
\hypersetup{pdfauthor=David R. Piegdon}
\hypersetup{pdftitle=hacking in physically addressable memory}

\begin{document}

\pagestyle{empty}

\begin{titlepage}

\newpage

%----
\newlength{\centeroffset}
\setlength{\centeroffset}{-0.5\oddsidemargin}
\addtolength{\centeroffset}{0.5\evensidemargin}
\thispagestyle{empty}
\vspace*{\stretch{1}}
\noindent\hspace*{\centeroffset}

\makebox[0pt][l]{
	\begin{minipage}{\textwidth}
		\flushright
			Rheinisch-Westf\"alische Technische Hochschule Aachen \\
			Lehrstuhl f\"ur Informatik IV\\[6ex]
			Seminar of Advanced Exploitation Techniques, WS 2006/2007\\[12ex]

			{\huge\bfseries hacking in physically addressable memory}\\[0ex]
			\noindent\rule[-1ex]{\textwidth}{3pt}\\[1ex]
			\hfill\emph{\Large a proof of concept}\\[1.5ex]

	\end{minipage}}
	\vspace{\stretch{1}
}

\noindent\hspace*{\centeroffset}
\makebox[0pt][l]{
	\begin{minipage}{\textwidth}
		\flushright
			{\bfseries by \textit{David R. Piegdon}\\[1.5ex] }
%			\today\\[3ex]
			April 12, 2007\\[3ex]
			Supervisor: Lexi Pimenidis
	\end{minipage}
}

\vspace{\stretch{2}}

\makebox[0pt][l]{
	\begin{minipage}{\textwidth}
		\flushright
%			\epsfysize 2cm
%			\epsfbox{logo-firewire.eps}
			\epsfysize 2cm
			\epsfbox{matrix-homepage.eps}
	\end{minipage}
}

%\pagebreak

\end{titlepage}

\begin{small}

%
% this is the second page...
\ 

\vspace{\stretch{1}}

The author, student of computer science at the \emph{Rheinisch-Westf\"alische
Technische Hochschule Aachen}, may be reached via email. To obtain the full
source code, this paper or other information, please refer to the following
link.

\begin{center}

	\textit{David Rasmus Piegdon} \\
	\href{mailto:david.rasmus.piegdon@rwth-aachen.de}{david.rasmus.piegdon@rwth-aachen.de} \\
	\href{http://david.piegdon.de/products.html}{http://david.piegdon.de/products.html} \\

\end{center}


This text was written on \emph{Linux} with \emph{vim}, typesetted with \LaTeX\
and generated by an implementation of \emph{RFC2795}. Please send fruit to the
author. No bananas.

\end{small}

% vim: tw=80 ai fdl=99 fo+=a
%
% $Id$
%

\section{Abstract}

When accessing memory via physical addressing, one will miss all the features
provided by a CPU to a process that accesses the same memory: all kinds of
address translations, caching and alike. This paper will give an overview
of these problems in general and then, mostly limited to the IA32 architecture
running a linux 2.6 kernel with no more than 4GB RAM and
3GB userspace/1GB kernelspace mapping, discuss these in detail and give some
solutions.







\pagestyle{headings}

\tableofcontents

% anfang des textes:

%\pagestyle{headings}
%% vim: tw=80 ai fdl=99 fo+=a
%
% $Id$
%

\section{Abstract}

When accessing memory via physical addressing, one will miss all the features
provided by a CPU to a process that accesses the same memory: all kinds of
address translations, caching and alike. This paper will give an overview
of these problems in general and then, mostly limited to the IA32 architecture
running a linux 2.6 kernel with no more than 4GB RAM and
3GB userspace/1GB kernelspace mapping, discuss these in detail and give some
solutions.







\newpage

% vim: tw=80 ai fdl=99 fo+=a
%
% $Id$
%

\section{Introduction}

Several advances in hacking via DMA will be introduces. As a foundation for the
attacks, \texttt{libphysical} and \texttt{liblinear} will be introduced to
access virtual address spaces of any process on the to-be-attacked host.



To have a simple, generic interface for all kinds of physical memory sources, we
implemented \texttt{libphysical}, where backends for new memory sources may be
plugged in. So far, it includes backends for IEEE1394 and filedescriptors.

In \textbf{section \ref{memsources}}, \texttt{libphysical} will be introduced
and different backends with their advantages and disadvantages will be
discussed.



Once, access to the physical memory of a system is obtained, there are two
obvious ways to extract useful information from it:

\begin{itemize}

	\item It is possible to parse the operating systems internal data
	structures holding all relevant information about loaded drivers,
	running processes et al.
	
	\item It is possible to use the information that the operating system
	provides to the hardware to tell it about the virtual address spaces of
	each process.
	
\end{itemize}

The first scenario will not work between different operating systems and
architectures, it will be necessary to write a parser for each combination of
them, possibly even for different versions of the same operating system.

The latter uses an information structure that only changes between different
architectures, as the architecture relies on it. Furthermore there is a well
defined algorithm for using this information (implemented in hardware in the
architecture, but well defined in the reference manuals for this architecture,
so system designers can provide valid data to the hardware).

On the other hand, the first approach will give much more information about the
system than the second, as we obtain all information directly from the kernel
structures, while using the second approach we only can enter virtual address
spaces of processes.

\cite{finding_digital_evidence_in_physical_memory:2006} is parsing
kernel-structs of windows and Linux kernels. In the following \emph{we} will use
the second approach for most attacks, as this seems to be more robust and
automatable.  As the above mentioned paper is about \emph{finding an attacker}
and not \emph{attacking}, the forensic personal does know about the
architecture, the operating system and version and can build a copy of the
system to test its tools, while an attacker only can guess the architecture and
operating system and needs more robust tools for his attacks.  (Obviously this
is only a short-term argument, as an attacker can also write such tools for
\emph{all} OS version and architecture combinations\ldots but the heck with
this...).



In \textbf{section \ref{addresstranslations}}, \texttt{liblinear}, an interface
to access virtual address spaces, will be introduced. It incorporates a backend
for IA-32\footnote{This backend is missing algorithms for less-used
operation-modes of the IA-32 architecture, but it will work at least for most
kinds of the \emph{Linux} kernels ($\leq$ 4GB RAM). It has not yet been tested
with \emph{Windows} or \emph{MacOS X} and is missing features (Virtual-8086
mode) to work with DOS-processes running inside Windows.} and functions to find
virtual address spaces.



In \textbf{section \ref{attacks}}, several attacks will be introduces, ranging
from simple information gathering up to obtaining an interactive shell via DMA
only. 



In \textbf{section \ref{prospects}}, prospects will be given, what further kinds
of attacks seem to be possible and/or may be of interest.



% vim: tw=80 ai fdl=99 fo+=a
%
% $Id$
%

\section{Physically addressable memory sources: \texttt{libphysical}}

\label{memsources}

This section presents a simple, generic interface for all kinds of physical
memory sources. We implemented a simple interface, where backends for new memory
sources may be plugged in. This interface is called \texttt{libphysical} and
currently includes backends for IEEE1394 and file descriptors.

Modern computer hardware provides many protection and memory management
mechanisms in hardware. This includes hardware to provide a virtual address
space for each process, protection mechanisms to restrict a process to its own
resources only, paging to extend memory to harddisks and fragment available
memory, caching to access frequently used memory faster, and more. Obviously,
all these features are architecture and operating-system dependent. An
interested reader may read documentation on system programming (e.g\@.
\cite{rwth_syspro_scriptum:2002} or \cite{IA32_SDM_3a:2006,IA32_SDM_3b:2006}) to
obtain further information.

Assume a process with its virtual address space and its corresponding set of
pages. Each page in this virtual address space may be:

\begin{itemize}

	\item a real, physical memory page that is mapped into the virtual 
		address space, possibly cached in the CPU's cache,

	\item a used page that is swapped to other media, like a harddisk

	\item (depending on the operating system) a mapped buffer or file

	\item not used, and thus not mapped

\end{itemize}

Swapped pages will and mapped pages may be loaded only on demand (i.e\@. when
the process tries to access the page), as access to a non-mapped page by a
process will generate a page fault and the operating system then may map the
demanded page.  Access to completely unused pages, via this mechanism, will
create the well known segmentation fault.

When access to physically addressed memory is obtained, that is a set of pages,
each page may be a page of memory of a random process, a buffer page of a
process, a page used by the operating system (kernel code, kernel data, kernel
stack, IO buffer, \dots), an unused page or a page used to give the CPU
information on how to handle virtual addresses, as this is done in hardware.
The latter pages will be called address translation tables; for more information
on these, see section \ref{address_translation}.

\subsection{Swapping, Multiple Accessors, Caching, Address Translation}

%Virtual pages of a process may be swapped, buffers may only be created on
%demand, and pages may be cached in other memory.

As physical memory only gives access to pages that are mapped from this
physical memory, we will be unable to access swapped pages and buffers that
have not been mapped. There is no simple solution to access this data; it is
required to call special operating system routines to do this; but access to
physical memory does not include access to the CPU by itself, and these routines
may be different from operating system to operating system.

Depending on the method used to access the memory, a parallel accessor may be
using the same memory at the same time. E.g\@. when using firewire (see
\ref{phys1394}) to read a page of a currently running task, this task may
access, thus read or write this page at the same time. For instance, if both CPU
and another accessor write at the same moment to the same address location, it
depends on timing and caching, which write access will be performed first and
thus override the other one; as an external accessor we have no way of knowing,
when the systems CPU accesses a page and if it is cached, thus leaving us with
no simple means to determine the success of a writing operation.  Also, reading
and writing at the same time may be impossible via the given method; thus many
atomic commands used for process synchronization, like ``test and set'', will
not exist.  Caching may also prevent these.

If a page is cached in another, faster memory, a copy of it will typically
reside in physical memory. In most cases we will not know if a page is cached or
not; on IA-32 however the address translation tables contain a flag for each
page telling the CPU if it may be cached or not. Depending on the way used to
access the memory, it may circumvent the cache or not have access to it at all.
When accessing a page, changes made by a task running in parallel may not be
visible immediately and changes made by us may be invisible to a parallel task
or maybe even overwritten by the cache at any time. Special care needs to be
taken to minimize this risk. When writing to pages, pages should be chosen that
are not cached or unlikely to be cached while writing; when reading pages, it
must be taken into consideration that the data may change at any time or may
have changed yet.

On systems using paging, physical memory will mostly be a concatenation of
``random'' pages, each one either used by some process or the operating system.
A minor part of these pages will be address translation tables, telling the CPU
what the virtual address space of different processes looks like. Where these
pages are is only known to the operating system and the CPU.  For a detailed
discussion, see section \ref{address_translation}.



\subsection{IEEE1394}

\label{phys1394}
% 1394a vs 1394b ?
IEEE1394, also known as \epsfysize 0.3cm \epsfbox{logo-firewire-black.eps}
firewire (Apple) or iLink (Sony), is an extension bus available on many modern
computer systems and devices.  In contrast to USB, which is a serial periphery
bus, firewire is a high-speed serial expansion bus with features like guaranteed
bandwidth (which is of interest for many real-time applications, like media
crunching), DMA\footnote{Direct Memory Access} and the ability to connect
multiple nodes with a single firewire-bus.  The concept of bus master and bus
slaves, as known from USB, is only virtual.  Typically when plugging together a
firewire bus, a node is randomly selected to be the master and manages this bus.
Most of these nodes have the ability to be bus master.

DMA is implemented in hardware by the OHCI chip set; it is used to release the
CPU from I/O operations. Mechanisms of preventing unwanted access exist, but
many drivers do not activate these methods by default. In the case of windows,
the operating system can be tricked into giving DMA to any device (see section
\ref{windows-dma}).
%Up to 64 devices may be plugged into one bus. Each one will choose a bus-unique
%node ID $[0..63]$; the bus master will have the highest node ID. All node IDs
%will be sequential, starting with 0. 
Access to memory of a node will require a 10 bit field for the bus ID, a 6 bit
field for the node ID and a 48 bit address field.  On Linux,
\texttt{libraw1394}\footnote{\texttt{libraw1394}:
\href{http://www.linux1394.org/}{http://www.linux1394.org/}} provides an easy
and portable interface to access the memory of a node. %Ideally, the full
%physical memory is accessible via the 48 bit address field, mapped in this
%address space starting from \texttt{0x0}; around \texttt{0xffff~f000~0000}
%several control state registers (CSRs) are located that provide information
%about the given target node and the capabilities of the firewire device.

%To disable the DMA-feature of firewire completely, load the \emph{ohci1394}
%module with option \emph{phys\_dma=0} on Linux or set the security mode in
%open-firmware to something different than ``none'' for apples powerbooks and
%ibooks. 

For more information on the underlying hardware or protocols, please refer to
\cite{OHCIspecs:2000,fwire_sys_arch:2222} or the
\texttt{libraw1394}\footnotemark[\value{footnote}] documentation.


%\subsection{DMA over IEEE1394}

%Firewire can be used to gain full access to a computer's memory, with nothing
%more than simply plugging a cable into a computers firewire port. Besides a set
%of possibly generated messages in the system's logfile, no traces will remain
%on the target computer.

% FIXME blocksize 32bit, align 32 bit :: ist dann problem immer noch da? /FIXME

Using \texttt{libraw1394} under Linux, it is possible to read different block
sizes of data via firewire on the remote computer. Our experiments showed that
different hardware allows bigger blocks to be read at different addresses:
4~byte blocks should always work; 1024~byte blocks may be read with some
hardware, if the address resolves to the physical memory. Control state
registers are likely to be readable only in 4~byte blocks.

%A possible problem is that accessing the memory of a running system \emph{may}
%generate non maskable interrupts (NMI) on the target system and even leave the
%operating system\footnote{E.g\@. some instances of Linux kernel 2.6.17.} in a
%non-usable state.  There are also some laptops and workstations with on-board
%controllers being disturbed by this, sometimes resulting in crashes.  Computers
%with PCI extension card, though, just worked fine during all tests. It is
%possible that the deeply incorporated on-board controllers interfere with the
%CPU accessing the main memory. This needs to be investigated in future work.

\label{windows-dma} Windows XP does use OHCI features to implement protection
mechanisms to prevent arbitrary external devices from reading any memory
location.  Adam Boileau,``TMASKY'' and others have shown in
\cite{rux2k6firewire:2006} that, by pretending to be a device like an iPod,
which ``deserves'' DMA (in terms of marketdroid\footnote{see jargon-files,
\href{http://catb.org/esr/jargon/html/M/marketroid.html}
{http://catb.org/esr/jargon/html/M/marketroid.html}}-logik), it is possible to
circumvent this ``protection'' and to trick Windows into giving an attacker DMA.
This attack is as simple as reading an iPod's config rom from its CSR and using
\texttt{libraw1394}'s \texttt{raw1394\_update\_config\_rom()} to use the copy.
Adam Boileau has implemented a simple script to do this. We have written our own
tool in C using \texttt{libraw1394}, which can be downloaded on request.


\subsection{Filedescriptor: \texttt{/dev/mem}, memory dumps}

Another source for physical memory may be given to an attacker via a
filedescriptor. This filedescriptor may refer to a memory dump or the Linux
\texttt{/dev/mem} device. In case of a plain memory dump, many of the mentioned
problems lapse: no caching will be performed, no concurrent process will change
the dumped data. In the case of a filedescriptor referring to \texttt{/dev/mem},
other accessors will exist, as \texttt{/dev/mem} is referring to the systems
memory; caching on the other hand should not be a problem as we are not
circumventing any caching system (like the CPU), but using it directly.

%\subsubsection{Problems with \texttt{/dev/mem}, solutions}

% FIXME: GiB mit ~ trennen? (http://webdesign.crissov.de/Typographie#Zahlen)

%But \texttt{/dev/mem} on Linux has other limitations\label{kerneluserdivision}.
%On 32 bit systems, the virtual address space has a size of $2^{32}$ bytes,
%i.e\@. 4GiB.  The virtual address space of a Linux process is divided into a
%userspace-part, where code, libraries and stack are mapped, and a
%kernelspace-part, where the complete kernel, data structures and the
%kernel-stack of this process are located. The userspace thread will not be able
%to access the kernelspace pages; but when the userspace process calls a system
%call, the CPU will jump into a different protection level and execute the
%system call entry code, which is part of the kernel. At this time the kernel
%runs in the same virtual address space, just at a different protection level.
%Default kernel configs divide the 4GiB virtual address space into a 3GiB
%userspace-part (\texttt{0x0000~0000} - \texttt{0xbfff~ffff}) and a 1GiB
%kernelspace-part (\texttt{0xc000~0000 - 0xffff~ffff}).  Different kernel config
%options (\texttt{CONFIG\_VMSPLIT\_3G}, \texttt{CONFIG\_PAGE\_OFFSET}) can
%change the split ratio, but the splitting itself is a basic design decision and
%thus a requirement for Linux to work. In the upper kernelspace-part, the lower
%physical memory will be fully mapped (up to about 900MiB), the kernel will - in
%this case - be located at roughly \texttt{0xc100~0000}, i.e\@. at the physical
%address \texttt{0x0100~0000}. During kernel configuration, these 900MiB are
%called ``precious lowmem'', because it is the only memory accessible by the
%kernel in a simple manner. When allocating kernelspace structures - this
%includes all address translation tables \footnote{on IA-32, these are:
%pagedirectories, pagetables, local- and global descriptor tables}
%\label{linuxATTinlowmem} - the kernel will normally only use this lowmem.  The
%\texttt{/dev/mem} driver will access this lowmem, when a userspace process
%utilizes it. But if the process tries to read more than the mapped lowmem,
%\texttt{/dev/mem} will fail to return this, as it is not mapped in the process.
%Thus, when using \texttt{/dev/mem}, only approx\@.  900GiB lower physical RAM
%will be accessible.
%
%It may be possible to work around this restriction: As stated, all address
%translation tables are located in the accessible memory. A process could try to
%find its own pagetable and manipulate it to map other regions of physical memory
%into its userspace section; then, no further access to \texttt{/dev/mem} would
%be required. As the reader will see in section \ref{findingATT}, identifying
%processes in Linux can mostly be as easy as looking at the upper userspace stack
%page.  Research on a system with 1.5GiB RAM has shown that most of these pages
%will be in the unmapped area, thus many of the processes will not be
%identifiable by this method. To make its own pagedirectory identifiable from
%the others, a process may e.g\@. map a random file at a typically never used
%address (e.g\@.  \texttt{0x6000~0000}), make sure the mapping was successful,
%access this file (to prevent a missing mapping-on-demand) and then search for
%the signature in all found pagedirectories. When found, it may map this
%pagedirectory at a special location and directly manipulate it. It should be
%taken care of the problem that a manipulated pagedirectory will only be
%reloaded into the CPU after a re-scheduling of the process, but a simple
%\texttt{usleep()} should suffice. The author is not sure, under which
%circumstances the Linux kernel does manipulate a process's pagedirectory; but
%obvious things like mapping new regions or unmapping mapped regions by system
%calls should be avoided, as the system may overwrite the manipulated pagetable
%with a new, adjusted one.



\subsection{Other sources}

The ideas described in this paper should be easily adoptable to all memory
sources giving access to physically addressable memory, this may include e.g\@.
remote management cards, suspend-2-disk images or virtual machines that have an
interface to access the virtual machines` memory.  \texttt{qemu} is such a
virtual machine, providing a \texttt{gdb} remote stub to attach a debugger.

To use a new physical source with the methods introduced in the later sections,
it is only required to write a new backend for \texttt{libphysical}.



% vim: tw=80 ai fdl=99 fo+=a
%
% $Id$
%

\section{Address translations; virtual, logical, linear and physical addressing}
\label{address_translation}

All multitasking environments that fulfill current requirements have to provide
virtual address spaces for each running process or thread. For performance and
security reason, this address translation from a process's virtual address to an
address valid in physical memory, this is normally performed in hardware.  These
mechanisms can include e.g. segmentation and paging.

A normal process's memory is divisible into several blocks or \emph{segments}:
the \emph{code segment} contains all the code that may be run; the \emph{data
segment} contains the static data that is known at compile time, global
structures or deliberately allocated memory; the \emph{stack segment} contains
the stack, including local variables.  On some architectures, it is possible to
assign segment descriptors, reffering to defined memory regions, to segment
registers.  This assignment will influence the further behaviour of address
translation: all addresses will from there on be taken to be relative to the
bound of the memory region specified by the segment descriptor.

\emph{Paging} will divide the virtual address space of a process into several
consecutive \emph{frames} of a specified page-size (typically 4096 bytes).
Virtual addresses can be split into a frame number and a frame offset; the frame
number is translated (mapped) via a translation table into a physical page
number and the frame offset is used as an offset into this physical page. If a
frame does not have a corresponding physical page, is is called to be unmapped.
Unmapped pages can be non-existing pages or can e.g, be swapped to slower media
like harddisks.

For a detailed description and discussion of these two important mechanisms, the
reader may reffer to a course on system programming, e.g.
\cite{rwth_syspro_scriptum:2002}.

On the IA32 architecture, the CPU can run in various modes of operation; for
secure multitasking operating systems, the \emph{protected mode} is the
preferred one.  The protected mode can use a two-level address translation:
first it will translate the \emph {logical address}, consisting of a segment
selector (which is an index into either the local or the global segment
descriptor table) and an offset to the \emph{linear address}.  The linear
address is then translated via paging to the \emph{physical address}. (The
paging translation is optional and needs to be enabled by setting a special flag
in a control register of the CPU.)

A linux process runs in a simple 4GiB flat virtual address space; no
segmentation is required. Thus, linux will create, among others, that are not of
interest for us, four special segments during boot-up: for each privilige level
(i.e. kernelspace and userspace), it will create segments for both code and
data. These four so called \emph{flat} segments will span the full virtual
address space of 4GiB, thus effectively elliminating segmentation. The address
of the \emph{global descriptor table}, holding the description of these
segments, is then loaded into the \emph{global descriptor table register} (GDTR)
and the specific segment registers are loaded with segment selectors refering to
the segments\footnote{This initialization is done in
\texttt{linux/arch/i386/kernel/head.S}, GDTs are defined at symbols
\texttt{boot\_gdt\_table} and \texttt{cpu\_gdt\_table}}.

The IA32 architecture divides the 4GiB virtual address space into 1024
4MiB-frames. This splitting is defined by the \emph{pagedirectory}. Each entry
of a pagedirectory is 4 Bytes long, thus the pagedirectory is $4*1024 = 4096$
Bytes long. Each of these \emph{pagedirectory entries} (PDEs), if present (its
\texttt{PRESENT}-flag is set), can either refer to a 4MiB physical page or a
pagetable dividing this virtual 4MiB frame further into 4KiB frames. A
\emph{pagetable} is consisting of 1024 4-byte \emph{pagetable entries} (PTEs),
each corresponding to a 4KiB frame.

...

REQ: PDE vs. PAE ; no HIGHMEM-64G (?)

\subsection{Finding address translation tables}
\label{findingATT}

\subsubsection{OS and architecture dependencies; typical address space layout}

\subsubsection{Matching via statisticts: NCD (nearest compression distance)}



\subsection{Example implementation: ia32 backend for \texttt{liblinear}}





% vim: tw=80 ai fdl=99 fo+=a
%
% $Id$
%

\section{Attacking}

\label{attacks}



\subsection{Information Gathering}

\subsubsection{Identifying processes}

\label{identifying_processes}  Once an address translation table has been found,
it is of interest, what kind of process resides in this virtual address space.
For userspace applications on IA-32-Linux there is a simple way to identify a
process's \emph{filename}, its \emph{arguments} and even its full set of
\emph{environment variables}: This information is often required by a process
and thus the kernel will provide it to the process by copying it to the bottom
pages of the application's stack\footnote{i.e.  the stack-pages that are found
first when seeking downward from virtual address \texttt{0xbfff~f000}}.

\texttt{proc\_info()} will seek the stack-bottom, parse it and return
ready-for-use environment vectors, command-line vectors and the full path of the
binary for a given linear address space. \texttt{remote-ps}, located in
\texttt{attacks/information/}, uses \texttt{proc\_info()} for each found address
space and will print a list of all found processes with its arguments.

% FIXME ``etwas mehr drueber, was man damit alles boeses machen kann''

\subsubsection{Typical places to find secrets}

Many applications keep secrets in their memory, some of them even locking them
into the main memory\footnote{e.g. via the \emph{mlock} function} to prevent the
operating system from swapping them to slower (permanent) media. While in
general this is a good idea, as an attacker may reconstruct the data from e.g.
thrown-away harddisks, it increases the chance of an attacker that can obtain
access to the memory of the system in question, as the secret material will be
stored in memory completely and unfragmented

``Secrets'' includes, among other information, \emph{authentication data},
\emph{cryptographic key material}, \emph{random data} (e.g. to seed a
cryptographic algorithm) and sometimes even \emph{algorithms} (\emph{proprietary
software}). Authentication data can be e.g.  passwords or private keys for
signature algorithms. Cryptographic key material, as the name says, are keys for
usage with cryptographic algorithms (like the above signature algorithms). These
two will be of main interest in this section.

Many applications using a cryptographic infrastructure for communications will
keep once loaded passwords or keys in their main memory for successive usage.
The operating systems protection model ensures the safety of this information
from other processes running on the same system; but by accessing the main
memory we do have full access to this material. The only task that remains is to
reconstruct the key material and passwords from the memory.

As an example, the following applications are of interest:

\begin{itemize}

	\item GnuPG and PGP: applications to sign and encrypt arbitrary data
	with public/private keypairs. They are wide spread for email-encryption
	and -signing.

	\item sshd, ssh and ssh-agent: the \emph{secure shell} application is an
	extended, encrypted version of telnet using strong cryptography,
	including passwords, skey, x509 certificates, RSA and DSA keys.

	\item Apache and other SSL-enabled web servers.

	\item OpenVPN, Cisco-VPN and other VPN-servers and clients

	\item Instant Messaging Applications, e.g. Psi keeps the authentication
	information and possibly the GnuPG keypair in memory.

	\item The computer BIOS password, ATA password or PGP-Wholedisk
		password: the computer or its drives can be locked with a BIOS
		password or the harddisk can be encrypted.  For a sample attack,
		see \cite{rux2k6firewire:2006}.

\end{itemize}


\subsubsection{Example attack: ssh-agent snarfer}

\label{ssh-agent-snarfer} To proof that it is easy to obtain secret keys from a
process we have written a sample attack to obtain
%
(snarf\footnote{to snarf: To grab, esp. to grab a large document or file for the
purpose of using it with or without the author's permission. // To acquire, with
little concern for legal forms or politesse (but not quite by stealing).
(source: \href{http://catb.org/jargon/html/S/snarf.html}{Jargon Files})})
%
\emph{ssh public/private keypairs} from \texttt{ssh-agent}s via \emph{firewire}.

When using \texttt{ssh} for accessing remote computers it is possible to
authenticate via passwords, public/private keypairs and various other methods.
The usage of public/private keypairs is wide-spread among people using
\texttt{ssh} on a regular basis. These keypairs can either be a DSA or a RSA
keypair, they are typically created with \texttt{ssh-keygen} and stored
somewhere in \texttt{\$HOME/.ssh/}, e.g.  \texttt{/root/.ssh/id\_dsa} and
\texttt{/root/.ssh/id\_dsa.pub}. Keypairs can and should be encrypted with a
passphrase to prevent attackers from using them, if they were able to obtain
them somehow. Thus to use a keypair it is required to enter this passphrase each
time. This can be disturbing during frequent usage, e.g. when using
\texttt{ssh+svn} or \texttt{scp} with remote-tab-completion (\texttt{zsh} is
capable of this).

For these and other reasons, the \texttt{ssh-agent} has been developed. This
agent will run in the background; the user can store a keypair into it (once
entering the passphrase to unlock the keypair) and successively use the keypair
without the requirement to enter the passphrase each time. The keypair can be
wiped from memory on demand and also be loaded only for a specified period of
time.

%Actually, here begins the fun.
During our tests we found that the key is \emph{not} wiped from memory when the
time limit is hit. It will be wiped the next time the ssh-agent is queried (via
its socket), but the agent is stalled in a \emph{read} system call until this
query and thus can not wipe the key. That makes it possible to obtain long
overdue keys from \texttt{ssh-agent}s, although their owners believed them to be
safe. A simple timer could have prevented this\footnote{We hereby strongly
encourage the developers to implement such a timer!}. But even with such a timer
enabled it would be possible to acquire the key during its lifetime.

To obtain a keypair from an agent via firewire, a staged attack is required:

%\lstset{numbers=left, numberstyle=\tiny, stepnumber=2, numbersep=5pt}
\lstset{language=C, numbers=left, numberstyle=\tiny, frame=lines}

\begin{figure}[ht] \begin{center}
	\tiny
	\begin{lstlisting}
		typedef struct identity {
			TAILQ_ENTRY(identity) next;
			Key *key;
			char *comment;
			u_int death;
			u_int confirm;
		} Identity;

		typedef struct {
			int nentries;
			TAILQ_HEAD(idqueue, identity) idlist;
		} Idtab;

		/* private key table, one per protocol version */
		Idtab idtable[3];
	\end{lstlisting}
	\caption{\texttt{openssh/ssh-agent.c}: Identity structure and \texttt{idtable}}
	\label{fig:code:identity-struct}
\end{center}\end{figure}

\begin{figure}[ht] \begin{center}
	\tiny
	\begin{lstlisting}
		enum types {
			KEY_RSA1,
			KEY_RSA,
			KEY_DSA,
			KEY_UNSPEC
		};

		struct Key {
			int      type;
			int      flags;
			RSA     *rsa;
			DSA     *dsa;
		};
	\end{lstlisting}
	\caption{\texttt{openssh/key.h}: Key-structure}
	\label{fig:code:key-struct}
\end{center}\end{figure}

\begin{figure}[ht] \begin{center}
	\tiny
	\begin{lstlisting}
		/* Declared already in ossl_typ.h */
		/* typedef struct rsa_st RSA; */
		struct rsa_st {
			int pad;
			long version;
			const RSA_METHOD *meth;
			ENGINE *engine;
			BIGNUM *n;
			BIGNUM *e;
			BIGNUM *d;
			BIGNUM *p;
			BIGNUM *q;
			BIGNUM *dmp1;
			BIGNUM *dmq1;
			BIGNUM *iqmp;
			int flags;
		};
	\end{lstlisting}
	\caption{\texttt{openssl/crypto/rsa/rsa.h}: RSA structure (stripped down)}
	\label{fig:code:rsa-struct}
\end{center}\end{figure}

\begin{figure}[ht] \begin{center}
	\tiny
	\begin{lstlisting}
		/* Already defined in ossl_typ.h */
		/* typedef struct dsa_st DSA; */
		struct dsa_st {
			int pad;
			long version;
			int write_params;
			BIGNUM *p;
			BIGNUM *q;      /* == 20 */
			BIGNUM *g;
			BIGNUM *pub_key;  /* y public key */
			BIGNUM *priv_key; /* x private key */
			int flags;
			const DSA_METHOD *meth;
			ENGINE *engine;
		};
	\end{lstlisting}
	\caption{\texttt{openssl/crypto/dsa/dsa.h}: DSA structure (stripped down)}
	\label{fig:code:dsa-struct}
\end{center}\end{figure}

\begin{figure}[ht] \begin{center}
	\tiny
	\begin{lstlisting}
		/* Already declared in ossl_typ.h */
		/* typedef struct bignum_st BIGNUM; */
		struct bignum_st {
			BN_ULONG *d;    /* Pointer to an array of 'BN_BITS2' bit chunks. */
			int top;        /* Index of last used d +1. */
			/* The next are internal book keeping for bn_expand. */
			int dmax;       /* Size of the d array. */
			int neg;        /* one if the number is negative */
			int flags;
		};
	\end{lstlisting}
	\caption{\texttt{openssl/crypto/bn/bn.h}: BIGNUM structure (stripped down)}
	\label{fig:code:bignum-struct}
\end{center}\end{figure}


\begin{enumerate}

	\item Seek the first GiB of physical memory for pagetables.

	\item For each pagetable: check with the introduced
		\texttt{proc\_info()}, if the found userspace belongs to a
		\texttt{ssh-agent} process. If not, seek next pagetable.

	\item Use the obtained environment to resolve the users home directory
		(\texttt{\$HOME}) and create a path where keypairs most likely
		reside in the file system (e.g.  ``\texttt{\$HOME/.ssh/}'') and
		seek this string in the heap.  This approach will only find
		keypairs that have been loaded with this
		key-location.\footnote{Actually this field is the key's
		comment-field that is mostly unused and overwritten with the
		filename of the key. Keypairs that are used with SSH protocol
		version 2 (virtually all) do not have a comment-field; during
		loading, the comment-field is always initialized with the keys
		pathname.} Keypairs loaded from different locations or via a
		relative path can thus not be found by this search.
	
	\item All loaded keypairs have a corresponding \emph{identity-struct} in
		an agent (see figure \ref{fig:code:identity-struct}). Among
		other fields, this identity-struct contains a link to a key
		struct, the above mentioned path/comment-field and the lifetime
		of the key. Thus to find the identity struct corresponding to a
		found comment-field, one has to search the address of the
		comment-field in the heap of the agent.
	
	\item Once the \emph{key-struct} (figure \ref{fig:code:key-struct}),
		that is linked to by the identity-struct, has been found, one
		can determine whether the found key is a RSA or a DSA key.  The
		key-struct contains a type-field and two pointers to either the
		RSA or the DSA key.  These referenced structures are the
		\emph{OpenSSL}\footnote{OpenSSL
		(\href{http://openssl.org}{http://openssl.org}) is a free
		open-source implementation of the secure socket layer protocol
		also providing a general purpose cryptography library
		(\text{libcrypto}).}-structures \texttt{RSA} and \texttt{DSA}.
	
	\item For both RSA and DSA structures (figure \ref{fig:code:rsa-struct}
		and \ref{fig:code:dsa-struct}), all important fields need to be
		recovered to obtain valid keypairs.  \cite{applied_crypto:1996,
		handbook_applied_crypto:2001} give an overview of both
		cryptographic algorithms, \cite{openssl_book:2002} introduces
		OpenSSL concepts and implementation details. OpenSSL`s arbitrary
		precision integer implementation is the \texttt{BIGNUM}-struct
		(often abbreviated ``BN'').  It consists of a variable-length
		array of bit-vectors forming the value and a length-field
		defining the length of this array (see figure
		\ref{fig:code:bignum-struct}).  As RSA and DSA both operate on
		finite fields, both are implemented with \texttt{BIGNUM}s.
		Therefore, the RSA and DSA structures contain several
		\texttt{BIGNUM}s that need to be recovered to obtain a valid
		copy of the keypair.

	\item Some validity tests may be done to verify that the acquired
		\texttt{BIGNUM}s fulfill algorithm-specific
		properties\footnote{During development it was useful to test
		obtained keys for some properties that are required. These tests
		are implemented in
		\texttt{attacks/information/sshkey-sanitychecks.c}} and thus
		form a valid keypair.

	\item Attach the obtained \texttt{BIGNUM}s back into valid RSA or DSA
		structures and save these keys to a file using
		openssl-functions.
	
\end{enumerate}
	
As seen, the search algorithm (2,3,4) has its downsides but works astonishingly
well\footnote{To create a better algorithm remains as an exercise to the
interested reader.}. A much better algorithm to find the \emph{identity}-structs
is to use the ELF-headers of the mapped executable to resolve the symbol of the
\emph{idtable}\footnote{The \emph{idtable} is a structure referencing all keys
that are loaded into the agent.}. This approach is straight-forward, hits
\emph{all} keys and should work almost always. The downside of this approach is
that most distributions distribute programs with their symbols stripped (due to
size and security reasons); this invalidates the symbol-resolution-approach, as
this stripping also removes any information of the \emph{idtable} symbol.

Furthermore once \emph{one} \emph{identity}-struct is found, \emph{all} structs
of the same key-type (RSA or DSA) could be found by walking the list this key is
linked into.

As stated above, the keypairs reside decrypted in the memory of the agent (even
if overtime) and thus, when snarfed and stored to a file, can be immediately
used by the command \texttt{ssh~-i~keyfile~user@host}~.  Such an attack will not
take much longer than searching the first 1 GiB of physical RAM for
pagedirectories, that is \emph{typically no more than 15 seconds}. If an attack
fails but an agent was found, it would be possible to just dump the heap of the
agent and stage a more thorough attack at a later time. Once the heap is dumped,
all required data is obtained. A similar attack via \emph{ptrace} should be
possible as well.

The reader may refer to \texttt{attacks/information/snarf-sshkey.c} in the
corresponding tarball for the source-code of the attack. Please keep in mind
that this attack will only find keys loaded with the absolute path
\texttt{\$HOME/.ssh/}.


\subsubsection{Matching and statistics to find secret keys}

\cite{hide_n_seek:1998} introduces some schemes to find secret keys in random
data and some countermeasures. It takes a special look at finding private RSA
keys if their corresponding public keys are known and finding keys by searching
high-entropy regions. Though we encourage the reader to read this interesting
paper, the circumstances are most likely very different now: cheap and small
storage media like flash-memory and small harddisks have increased portable
storage to a huge size, equal or larger than the memory a computer system
typically has. Thus, an attacker can just dump the full memory or a subset of it
(like the virtual address space of a single process) that is promising to
contain a secret. A thorough attack can then be staged later.  Still, searching
private keys with the introduced methods can be very helpful, if reconstruction
of the used data-structures is impossible or more expensive. Furthermore, when
trying to obtain a private key, often enough the corresponding public key is
unknown. This invalidates the approaches introduced to find RSA secret keys that
require the public key.




\subsection{Userspace Modifications}

It is important to know a few details about executables, libraries and processes
to understand this section.

Each executable object, including libraries and executables, can be separated
into code and data, where the code should be read-only during execution. The
data can further be separated into read-only data, read-write data and
uninitialized global variables (local variables will be allocated from the
stack, runtime-allocated memory is allocated from the heap).  Thus an executable
object may be split into four regions: \emph{code}, \emph{rodata} (constants),
\emph{rwdata} and \emph{dynamic data} (rwdata may also be implemented by copying
the initial data from rodata to dynamic data); stack and heap are
process-specific.

As many different processes may use the same executable objects, it would be a
waste of memory if the operating system created a copy of the object for each
reference to it.

Code and rodata may not be written to by a process, thus the operating system
can share these two regions among processes that are using the same objects
(binaries or libraries).  Thus, once an operating system \emph{ensures} that a
process can not write to code-regions and read-only data-regions, it can map
these once loaded regions into multiple virtual address spaces. This enforcement
is done in hardware, on IA-32 by setting a flag in the page directory or a page
table referencing the specific physical memory pages containing the region.

Newer CPUs provide \emph{page-level no-execute enforcements} (AMD's NX \emph{No
eXecute bit}, Intel's EXB \emph{EXecute disable Bit}); equal segment-level
enforcements exist for years but never have been used in mainstream as Linux and
many other operating-systems use a flat memory-model (with only one big segment
spanning the full virtual address space).  Once these page-level enforcements
are used in systems, attacks that inject code into data-regions or the stack are
rendered completely useless. However with access to the data-structures (page
directory and page tables) containing the information, which pages are
executable and which not, one can change this information before injecting code
e.g. into an applications stack. 

On Linux, programs and libraries are in \emph{Executable and Linker Format}
(ELF).  This format is described in the manpage \texttt{elf(5)}. When a binary
is mapped into a process's memory, it is mapped including the full ELF header
containing all information that is required to link all references between
different objects; ELFs are always mapped at page-bounds. Due to this, all
mapped ELFs (that includes executables and libraries) can be found by scanning
all pages for the ELF Magic (0x7f E L F) at offset $0$ in the page.  Libraries,
executables and other ELF objects can be distinguished by evaluating the
\texttt{e\_type} field of the ELF header.

\subsubsection{Overwriting executable or library code}

When code of executables or libraries is changed, all programs using these ELF
objects are influenced at the same time. An attacker thus has the ability, but
also the burden, to possibly infect several processes at the same time. Such an
attack has to be carefully prepared and conducted, as each system may have a
different version of a binary and overwriting the wrong parts of an ELF or
writing the wrong code may result in an almost immediate crash of all processes
using the ELF. Though this is a rather interesting approach, there are easier
ways to inject code into a \emph{single} process (see \ref{overwriting_stack}).

Such an attack could be conducted by searching a single virtual address space
for the glibc and then parsing the ELF-headers and searching for the entry-point
of the \texttt{printf}-function (or some other, less frequently used function).
Then a piece of code could be injected into the unused fragment of the last page
of the libc-mapping. It is important to inject the shellcode into the mapping,
as all processes that will be affected have to be able to reach the shellcode.
The intention is to overwrite the \texttt{printf} functions code with a
\emph{relative}\footnote{An ELF may be mapped at different locations in
different processes, if it is ``PIC'' or ``PIE'' (Position Independent
Code/Executable) and the kernel supports this.  Thus, unless the ELF is only
mapped in one process or the overwritten function is only used in one process,
the jumps target has to be addressed relative to the current position.} jump to
this injected code.  But as we need to overwrite some instructions inside the
function, we need to parse the functions code to separate each
instruction\footnote{on IA-32, different machine instructions can have different
length}, so that after our code is executed, it executes a copy of the
overwritten instructions and jumps to a fully intact instruction right after the
injected jump. After this has been done, the jump can be injected into the
functions code. This last write has to be as atomic as possible, as a process
may just be executing these bytes and thus get astray.  On IA-32, entry-points
of functions are most likely aligned to 32-bit addresses\footnote{due to
optimizations by the compiler}.  Firewire also provides an interface to write
32-bit aligned 32-bit values (``quads'') atomically.  Unfortunately, a
\emph{relative short} jump ($2$ bytes) can only jump within $\pm256$ bytes from
the jump itself and \emph{relative long} and \emph{absolute} jumps are $5$ bytes
wide ($1$ byte command + $4$ bytes address).  A short jump is most likely
incapable of reaching the last page of the ELF and writing a long relative jump
is not atomic.

A lot of interesting methods to inject code into a running process have been
developed; e.g.  \cite{phrack59.8:2002} gives an introduction into using
\emph{ptrace}, including injecting whole shared objects using the runtime linker
\emph{libdl}.  The usability of this approach has not yet been analysed.


\subsubsection{Overwriting the stack and return addresses}

\label{overwriting_stack} Besides stealing SSH-keys, we have put most of our
efforts into injecting code into the stack and overwriting return-addresses on
the stack to point to the injected code. This modification of the classic
\emph{stack overflow} method has some advantages over the previous approach:

\begin{itemize}

	\item Each process, even each thread, has its own stack. Thus only a
		single thread will be affected by the attack.

	\item If the attack fails and the thread dies, only a \emph{single}
		thread will fail on the target system, not e.g. \emph{all}
		processes using the glibc or \emph{all} instances of $/bin/sh$.

	\item The process can read and write to the stack as well, thus we can
		communicate with the injected shell-code in a rather easy way
		(see \ref{communication_DMA}).

	\item During the attack we do not need to modify parts of the code of
		the target-process, reducing the risk of an astray process. The
		final part consists of overwriting 4 byte wide return-addresses
		on the stack and this can almost always be done automatically.

\end{itemize}

The attack consists of the following steps:

\begin{enumerate}

	\item Search a free location in the stack-pages. If the shellcode is
		small, we can use the zero-padded area of the pages containing
		the environment and argument vectors (see section
		\ref{identifying_processes}). If the shellcode does not fit into
		this area, we could try to just overwrite these vectors. Most
		programs will parse environment and arguments only once during
		startup of the program, thus this should be rather safe.

	\item Scan the stack for stack-frames and for each found: overwrite the
		return address. This can be simplified into: overwrite all
		32-bit aligned 32-bit values that contain a value that might be
		a pointer into the main code area (\texttt{0x08**~****}, see
		figure \ref{fig:linux_virtual_address_space}) with a pointer to
		the injected code.  A more aggressive approach might also
		overwrite return-values pointing into the library section
		(\texttt{0xB7**~****}).

\end{enumerate}

Once the attacked process leaves a stack frame with an overwritten return value,
it will jump to the injected code and execute it.

An implementation of this attack, including some sample shellcodes like a
bindshell and a simple \texttt{printf} can be found in
\texttt{attacks/userspace/inject-code.c} and \linebreak
\texttt{attacks/userspace/shellcodes/}.

% FIXME: mehr infos zu ``3.1.1:Identifying processes'', danach hier darauf
% aufbauen. Mehr infos zu stack-frames?

\subsubsection{Direct communication with shellcode via DMA}

\label{communication_DMA} The masterpiece of shellcode-writing is a shellcode
that spawns an interactive shell (thus the name ``shellcode'') and yet is as
invisible on the target system as possible.  An interactive shell has to
communicate with its user, so typically a network-based bindshell is used for
this attack. The downside is the visibility of the communication on the
network-layer: an administrator can easily spot the network connection by either
sniffing on the network or by asking the system what kind of sockets and files a
process is using\footnote{e.g. by using \texttt{lsof -i} (LiSt Open Files)}. A
network intrusion system (NIS) can easily spot bindshell connections in an
automated way or firewalls could be configured in a way that a network
connection is impossible. When using e.g. firewire to attack a host it is even
possible that there is no network connection between the attacker and the victim
at all.

The bindshell approach has obvious disadvantages. Thus we will use the same
attack vector that has been used to inject the ``beachhead'': physical memory
access.

The overall mechanism is introduced in figure \ref{fig:functionality_beachhead}.
The injected shellcode will fork a shell and communicate with stdin/stdout of
the shell via two pipes. The shellcode then creates a second thread, then having
one thread for each direction of \emph{master to shell} and \emph{shell to
master}. If the master (attacker) wishes to send a command to the shell, it
writes the command string into the \emph{FromMaster} ring-buffer via DMA.  Once
the ReaderThread sees that the ring-buffer is not empty, it reads the data
inside the buffer and writes it into the pipe to the shell.  The WriterThread
will read data coming from the shell from the pipe and then write it into the
\emph{ToMaster} ring-buffer, so the master can read it via DMA.

The shellcode of the beachhead
(\texttt{attacks/userspace/shellcodes/dmashellcode.s}) is introduced in figure
\ref{fig:beachhead_pseudocode} in pseudo-code. The program to inject the
shellcode and communicate with it is \texttt{attacks/userspace/dmashell.c}.

\begin{figure}[ht] \begin{center}

	\epsfysize 6cm

	\epsffile{functionality_beachhead.eps}

	\caption{Functionality of the beachhead}

	\label{fig:functionality_beachhead}

\end{center}\end{figure}


\begin{figure}[htp] \begin{center} \tiny
	\begin{lstlisting}
		// ringbuffer for data from beachhead to master
		RingBuffer	ToMaster;
		// ringbuffer for data from master to beachhead
		RingBuffer	FromMaster;

		// pipe for data from beachhead to shell
		int		ToShell[2];
		// pipe for data from shell to beachhead
		int		FromShell[2];
		// flag, if pipes are still valid (volatile because two threads access it)
		volatile bool	pipes_ok = 1;

		int		child_pid;
		// set by master if child should be killed
		volatile bool	terminate_child = 0;
		// set by beachhead to signal death of child
		bool		child_is_dead = 0;
		// set by master to acknowledge death of child
		volatile bool	child_is_dead_ACK = 0;

		start:
			pipe(ToShell[]);
			pipe(FromShell[]);
			switch fork() {
				case child:
					dup2(ToShell[0], stdin);
					dup2(FromShell[1], stdout);
					dup2(FromShell[1], stderr);
					execve("/bin/sh");
					// on failure: exit.
					exit();
					;;
				case parent:
					child_pid = returnvalue of fork;
					close(ToShell[0]);
					close(FromShell[1]);
					// create a second thread "WriterThread"
					clone(WriterThread);
					// current thread becomes "ReaderThread"
					goto ReaderThread;
					;;
			}


		ReaderThread:
			// the ReaderThread will read from the master and relay to the shell
			while( pipes_ok ) {
				if(terminate_child) {
					// attacker requested shell to be terminated
					kill(child_pid, SIGKILL);
					pipes_ok = 0;
					// ReaderThread terminates, WriterThread will do cleanup
					exit();
				}
				if( isEmpty(FromMaster) ) {
					sleep(0.001 seconds);
				} else {
					if(1 != write(ToShell[1], FromMaster.buffer
							+FromMaster.currentLocation, 1 byte)) {
						// can not communicate with child
						pipes_ok = 0;
					} else
						FromMaster.currentLocation += 1;
				}
			}
			exit();

		WriterThread:
			// the WriterThread will read from the shell and relay to the master
			while( pipes_ok ) {
				if( isFull(ToMaster) ) {
					sleep(0.001 seconds);
				} else {
					if(1 != read(FromShell[0], ToMaster.buffer
							+ToMaster.currentLocation, 1 byte)) {
						// can not communicate with child
						pipes_ok = 0;
					} else
						ToMaster.currentLocation += 1;
				}
			}
			// pipes are closed. tell the master, but wait at most 2 seconds
			while(!chils_is_dead_ACK && child_is_dead <= 2) {
				child_is_dead++;
				sleep(1 second);
			}
			exit();
			

	\end{lstlisting}
	\caption{Beachhead shellcode (\texttt{dmashellcode.s}), pseudocode}
	\label{fig:beachhead_pseudocode}
\end{center} \end{figure}



% vim: tw=80 ai fdl=99 fo+=a
%
% $Id$
%

\section{Future prospects}

\label{prospects}

\subsection{Kernelspace Modifications}

The Linux character device \texttt{/dev/kmem} gives a process read and write
access to its virtual address space, including the kernel-space area.
\texttt{/dev/kmem} has not only been used for its legitimate purpose (i.e.
debugging the kernel) but also to inject code into the running kernel and
install kernel-based rootkits. \cite{phrack58.7:2001} describes an attack using
\texttt{/dev/kmem} to inject a new syscall-handler (a regularly used rootkit
technique).

\subsubsection{Emulating \texttt{/dev/kmem}}

As \texttt{/dev/kmem} gives a process access to its virtual address space
without any restrictions, it is yet emulated by \texttt{liblinear}, as
\texttt{liblinear} provides the same functionality, once it has been loaded with
a pagetable of a random process.  A pagetable of any process will include the
fully mapped kernel space (``lowmem'', see section \ref{linuxATTinlowmem} and
section \ref{ATTguess}).

On the other hand, some new problems come up, if the attacker can not access
CPU-registers like the IDT\footnote{Interrupt Descriptor Table}, but wishes to
know the location of the system call table. If it is necessary to resolve kernel
symbols and the to-be-attacked kernel is LKM\footnote{Loadable Kernel Module}
capable, an attacker could inject a special shellcode to resolve all required
symbols via \texttt{get\_kernel\_syms} or inject a LKM and let the kernel do the
job. If however an active approach is impossible or the kernel is not LKM
capable, statistic approaches might be necessary.

\subsection{Bootstrapping custom operating systems}

As an attacker has complete access to a systems memory, it is possible to take
over the system completely, reset it to a known state and boot a custom
operating system on it (and e.g. use firewire storage as the root-device for the
new system). A special bootloader would be required to do this and it might also
be necessary to reset the system and attached hardware in a special way,
depending on the operating system running on it before the approach.



% vim: tw=80 ai fdl=99 fo+=a
%
% $Id: conclusion.tex 349 2007-02-04 15:43:37Z lostrace $
%

\section{Conclusion}

\label{conclusion}

It has been shown that firewire and other DMA technology are a mature attack
vector having a serious impact on a systems security. DMA interfaces should
always be sealed or disabled if untrusted persons can access them; this
particularly includes laptops, as more and more of them are equipped with a tiny
firewire port.  Security ``solutions'' that deny DMA for some devices and allow
DMA for others should be tested very carefully, as these schemes may be tricked
by pretending to be a different, ``trusted'' device (see
\cite{rux2k6firewire:2006}).

Though most of the tools introduced are designed to attack a system,
\texttt{libphysical} and \texttt{liblinear} can also be used for forensic
purposes to analyse memory dumps (with the filedescriptor backend). The
statement ``There is little experience in reconstructing logical/virtual memory
from physical memory dumps'' from \cite{cansecwest_firewire:2005} is no longer
true: \texttt{liblinear} can be used to access virtual address spaces of each
process (independent of the operating system), e.g.~IDETECT (by Mariusz Burdach,
\cite{finding_digital_evidence_in_physical_memory:2006}) can be used to analyse
kernel data structures to obtain other information.



% vim: tw=80 ai fdl=99 fo+=a
%
% $Id$
%

\section{Acknowledgements}

I would like to thank Maximillian Dornseif, Christian N. Klein and Michael
Becher for the initial idea and research (\cite{cansecwest_firewire:2005}), the
latter also for personal conversations; the Aachen University of Technology
(RWTH Aachen, \href{http://www.rwth-aachen.de}{http://www.rwth-aachen.de}) and
Lexi Pimenidis for giving me the chance to investigate the idea and write this
paper; Timo Boettcher and Alexander Neumann for ideas and comments; Swantje
Staar for her support concerning the english language and the Chaos Computer
Club Cologne (\href{http://koeln.ccc.de}{http://koeln.ccc.de} for its support in
general concerning projects not limited to, but including this one.



%\newpage
%\begingroup
%\parindent 0pt
%\parskip 2ex
%\def\enotesize{\normalsize}
%\addcontentsline{toc}{section}{Notes}
%\theendnotes
%\endgroup

\newpage
\begingroup

\addcontentsline{toc}{section}{List of figures}
\listoffigures

\addcontentsline{toc}{section}{References}
\bibliographystyle{alpha} %acm or alpha
\bibliography{paper}

\endgroup

% ende

\end{document}

