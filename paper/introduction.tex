% vim: tw=80 ai fdl=99 fo+=a
%
% $Id$
%

\section{Introduction}

Several advances in hacking via DMA will be introduces. As a foundation for the
attacks, \texttt{libphysical} and \texttt{liblinear} will be introduced to
access virtual address spaces of any process on the to-be-attacked host.



To have a simple, generic interface for all kinds of physical memory sources, we
implemented \texttt{libphysical}, where backends for new memory sources may be
plugged in. It includes backends for IEEE1394 and filedescriptors so far.

In \textbf{section \ref{memsources}}, \texttt{libphysical} will be introduced
and different backends with their advantages and disadvantages will be
discussed.



Once, access to the physical memory of a system is obtained, there are two
obvious ways to extract useful information from it:

\begin{itemize}

	\item It is possible to parse the operating systems internal data
	structures holding all relevant information about loaded drivers,
	running processes et al.
	
	\item It is possible to use the information that the operating system
	provides to the hardware to tell it about the virtual address spaces of
	each process.
	
\end{itemize}

The first scenario will not work between different operating systems and
architectures, it will be necessary to write a parser for each combination of
them, possibly even for different versions of the same operating system.

The latter uses an information structure that only changes between different
architectures, as the architecture relies on it. Furthermore there is a well
defined algorithm for using this information (implemented in hardware in the
architecture, but well defined in the reference manuals for this architecture,
so system designers can provide valid data to the hardware).

On the other hand, the first approach will give much more information about the
system than the second, as we obtain all information directly from the kernel
structures, while using the second approach we only can enter virtual address
spaces of processes.

In \cite{finding_digital_evidence_in_physical_memory:2006}, an approach is
introduced that parses kernel-structs of windows and Linux kernels. In the
following \emph{we} will use the second approach for most attacks, as this seems
to be more robust and automatable.  As the above mentioned paper is about
\emph{finding an attacker} and not \emph{attacking}, the forensic personal does
know about the architecture, the operating system and version and can build a
copy of the system to test its tools, while an attacker only can guess the
architecture and operating system and needs more robust tools for his attacks.
(Obviously this is only a short-term argument, as an attacker can also write
such tools for \emph{all} OS version and architecture combinations\ldots but the
heck with this...).
% FIXME ``the heck with this''... spida: ``for simplicities sake, I will do this
% for only one one combination here, as this is intended to be only a proof of
% concept''



In \textbf{section \ref{addresstranslations}}, \texttt{liblinear}, an interface
to access virtual address spaces, will be introduced. It incorporates a backend
for IA-32\footnote{This backend is missing algorithms for less-used
operation-modes of the IA-32 architecture, but it will work at least for most
kinds of the \emph{Linux} kernels ($\leq$ 4GB RAM). It has not yet been tested
with \emph{Windows} or \emph{MacOS X} and is missing features (Virtual-8086
mode) to work with DOS-processes running inside Windows.} and functions to find
virtual address spaces.



In \textbf{section \ref{attacks}}, several attacks will be introduces, ranging
from simple information gathering up to obtaining an interactive shell via DMA
only. 



In \textbf{section \ref{prospects}}, prospects will be given, what further kinds
of attacks seem to be possible and/or may be of interest.

