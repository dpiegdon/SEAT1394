% vim: tw=80 ai fdl=99 fo+=a
%
% $Id$
%

\section{Conclusion}

Though some problems remain, it has been shown that firewire and other DMA
technology are a mature attack vector having a serious impact on a systems
security. DMA interfaces should always be sealed or disabled if untrusted
persons can access them; this particularly includes laptops, as more and more of
them are equipped with a tiny firewire port.  Security ``solutions'' that deny
DMA for some devices and allow DMA for others should be tested very carefully,
as these schemes may be fooled by pretending to be a different, ``trusted''
device (see \cite{rux2k6firewire:2006}). To disable the DMA-feature of firewire
completely, use ``\texttt{modprobe ohci1394 phys\_dma=0}'' on linux or set the
security mode in open-firmware to something different than ``none'' for apples
powerbooks and ibooks. 


Though most of the tools introduced are designed to attack a system,
\texttt{libphysical} and \texttt{liblinear} can also be used for forensic
purposes to analyse memory dumps (with the filedescriptor backend). The
statement ``There is little experience in reconstructing logical/virtual memory
from physical memory dumps'' from \cite{cansecwest_firewire:2005} is no longer
true: \texttt{liblinear} can be used to access virtual address spaces of each
process (independent of the operating system), e.g. IDETECT (by Mariusz Burdach,
\cite{finding_digital_evidence_in_physical_memory:2006}) can be used to analyse
kernel data structures to obtain other information.

