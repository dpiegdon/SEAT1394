% vim: tw=80 ai fdl=99 fo+=a
%
% $Id$
%

\section{Address translations; virtual, logical, linear and physical addressing}
\label{address_translation}

All multitasking environments that fulfill current requirements have to provide
virtual address spaces for each running process or thread. For performance and
security reason, this address translation from a process's virtual address to an
address valid in physical memory, this is normally performed in hardware.  These
mechanisms can include e.g. segmentation and paging.

A normal process's memory is divisible into several blocks or \emph{segments}:
the \emph{code segment} contains all the code that may be run; the \emph{data
segment} contains the static data that is known at compile time, global
structures or deliberately allocated memory (including the heap); the
\emph{stack segment} contains the stack, including local variables.  On some
architectures, it is possible to assign segment descriptors, reffering to
defined memory regions, to segment registers.  This assignment will influence
the further behaviour of address translation: all addresses will from there on
be taken to be relative to the bound of the memory region specified by the
segment descriptor.

\emph{Paging} will divide the virtual address space of a process into several
consecutive \emph{frames} of a specific page-size (typically 4096 bytes).
Virtual addresses can be split into a frame number and a frame offset; the frame
number is translated (mapped) via a translation table into a physical page
number and the frame offset is used as an offset into this physical page. If a
frame does not have a corresponding physical page, is is called to be unmapped.
Unmapped pages can be non-existing pages or can e.g, be swapped to slower media
like harddisks.

For a detailed description and discussion of these two important mechanisms, the
reader may reffer to a course on system programming, e.g.
\cite{rwth_syspro_scriptum:2002}.



\subsection{Example implementation: ia32 backend for \texttt{liblinear}}

On the IA32 architecture, the CPU can run in various modes of operation; for
secure multitasking operating systems, the \emph{protected mode} is the
preferred one.  The protected mode can use a two-level address translation:
first it will translate the \emph {logical address}, consisting of a segment
selector (which is an index into either the local or the global segment
descriptor table) and an offset to the \emph{linear address}.  The linear
address is then translated via paging to the \emph{physical address}. (The
paging translation is optional and needs to be enabled by setting a special flag
in a control register of the CPU.)

\label{linux_gdt} A linux process runs in a simple 4GiB flat virtual address
space; no segmentation is required. Thus, linux will create (among others, that
are not of interest for us) four special segments during boot-up: for each
privilige level (i.e. kernelspace and userspace), it will create segments for
both code and data. These four so called \emph{flat} segments will span the full
virtual address space of 4GiB, thus effectively elliminating segmentation. The
address of the \emph{global descriptor table}, holding the description of these
segments, is then loaded into the \emph{global descriptor table register} (GDTR)
and the specific segment registers are loaded with segment selectors refering to
the segments\footnote{This initialization is done in
\texttt{linux/arch/i386/kernel/head.S}, GDTs are defined at symbols
\texttt{boot\_gdt\_table} and \texttt{cpu\_gdt\_table}}.

The IA32 architecture divides the 4GiB virtual address space into 1024
4MiB-frames. This splitting is defined by the \emph{pagedirectory}. Each entry
of a pagedirectory is 4 bytes long, thus the pagedirectory is $4*1024 = 4096$
Bytes long. Each of these \emph{pagedirectory entries} (PDEs), if present (its
\texttt{PRESENT}-flag is set), can either refer to a 4MiB physical page or a
pagetable dividing this virtual 4MiB frame further into 4KiB frames. A
\emph{pagetable} is, again, consisting of 1024 4-byte \emph{pagetable entries}
(PTEs), each corresponding to a 4KiB frame.

REQ: PDE vs. PAE ; no HIGHMEM-64G (?)



\subsection{Finding address translation tables}
\label{findingATT}

When accessing a range of memory via physical addressing, it is neccessary to
find address translation tables to make sence out of the vast, unsorted number
of pages. Typically, translation tables are not marked as such and as we can not
access the processor or the operating system to ask it, where these are, we have
to search them. The following methods have proven themselves.



\subsubsection{OS and architecture dependencies; typical address space layout}

Obviously, address translation tables are architecture and operating system
specific; but within an architecture and an operating system, the will often
have shared data or specific patterns that are identifiable. For example, one
can omit searching the segment descriptor tables (see \ref{linux_gdt}) and
concentrate on finding pagedirectories. There are several special patterns that
can be found in a typical pagedirectory of a linux process running on IA32. To
understand, I will layout the typical address space of a linux process on IA32:

\begin{itemize}

	\item \emph{code and heap} will be starting somewhere aroung
	\texttt{0x0800 0000}, consecutively following with a minor number of
	unmapped pages between.

	\item \emph{libraries} will be mapped at \texttt{FIXME!}.

	\item the \emph{stack} will be located somewhere, possibly directly,
	below \texttt{0xC000 0000}. 

	\item from \texttt{0xC000 0000}, the approximately lower physical 970
	MiB of physical RAM will be mapped, only accessible when the accessing
	task is running in ring 0.

	\item the kernelspace stack and other kernel data structures are mapped
	around \texttt{0xF900 0000 (FIXME!)}.

	\item all unmapped pages will have 4-byte entries consisting of zeroes
	(\texttt{0x0000})

\end{itemize}

Stack- and memory randomization techniques like \emph{PaX} do randomize the base
addresses a bit, but the general layout stays the same.

Besides searching pages that show non-zero values around these positions and
zero values elsewhere, it is much easier and faster to just check, if the
virtual address \texttt{0xC000 0000} maps to the physical address \texttt{0x0},
because typically the PDE for page no. \texttt{0xC0000} will point to the 4MiB
physical page at \texttt{0x0}. This test only requires reading the 4byte PDE
entry \texttt{0xC00} and does sort out the vast majority of non-pagedirectory
pages.


\subsubsection{Matching via statisticts: NCD (nearest compression distance)}



