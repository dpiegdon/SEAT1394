% vim: tw=80 ai fdl=99 fo+=a
%
% $Id$
%

\section{Future prospects}

\subsection{Kernelspace Modifications}

The linux character device \texttt{/dev/kmem} gives a process read and write
access to its virtual address space, including the kernel-space area.
\texttt{/dev/kmem} has not only been used for its legitimate purpose (i.e.
debugging the kernel) but also to inject code into the running kernel and
install kernel-based rootkits. \cite{phrack58.7:2001} describes an attack using
\texttt{/dev/kmem} to inject a new syscall-handler (a regularly used rootkit
technique).

\subsubsection{Emulating \texttt{/dev/kmem}}

As \texttt{/dev/kmem} gives a process access to its virtual address space
without any restrictions, it is yet emulated by \texttt{liblinear}, as
\texttt{liblinear} provides the same functionality, once it has been loaded with
a pagetable of a random process.  A pagetable of any process will include the
fully mapped kernel space (``lowmem'', see section \ref{linuxATTinlowmem} and
section \ref{ATTguess}).

On the other hand, some new problems come up, if the attacker can not access
CPU-registers like the IDT\footnote{Interrupt Descriptor Table}, but wishes to
know the location of the system call table. If it is necessary to resolve kernel
symbols and the to-be-attacked kernel is LKM\footnote{Loadable Kernel Module}
capable, an attacker could inject a special shellcode to resolve all required
symbols via \texttt{get\_kernel\_syms} or inject a LKM and let the kernel do the
job. If however an active approach is impossible or the kernel is not LKM
capable, statistic approaches might be necessary.

\subsection{Bootstrapping custom operating systems}

As an attacker has complete access to a systems memory, it is possible to take
over the system completely, reset it to a known state and boot a custom
operating system on it (and e.g. use firewire storage as the root-device for the
new system). A special bootloader would be required to do this and it might also
be necessary to reset the system and attached hardware in a special way,
depending on the operating system running on it before the approach.

